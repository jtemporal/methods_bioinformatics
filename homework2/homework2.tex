\documentclass[]{article}
\usepackage{lmodern}
\usepackage{amssymb,amsmath}
\usepackage{ifxetex,ifluatex}
\usepackage{fixltx2e} % provides \textsubscript
\ifnum 0\ifxetex 1\fi\ifluatex 1\fi=0 % if pdftex
  \usepackage[T1]{fontenc}
  \usepackage[utf8]{inputenc}
\else % if luatex or xelatex
  \ifxetex
    \usepackage{mathspec}
  \else
    \usepackage{fontspec}
  \fi
  \defaultfontfeatures{Ligatures=TeX,Scale=MatchLowercase}
  \newcommand{\euro}{€}
\fi
% use upquote if available, for straight quotes in verbatim environments
\IfFileExists{upquote.sty}{\usepackage{upquote}}{}
% use microtype if available
\IfFileExists{microtype.sty}{%
\usepackage{microtype}
\UseMicrotypeSet[protrusion]{basicmath} % disable protrusion for tt fonts
}{}
\usepackage[margin=1in]{geometry}
\usepackage{hyperref}
\PassOptionsToPackage{usenames,dvipsnames}{color} % color is loaded by hyperref
\hypersetup{unicode=true,
            pdftitle={Homework 2},
            pdfauthor={Jessica Temporal 7547611},
            pdfborder={0 0 0},
            breaklinks=true}
\urlstyle{same}  % don't use monospace font for urls
\usepackage{color}
\usepackage{fancyvrb}
\newcommand{\VerbBar}{|}
\newcommand{\VERB}{\Verb[commandchars=\\\{\}]}
\DefineVerbatimEnvironment{Highlighting}{Verbatim}{commandchars=\\\{\}}
% Add ',fontsize=\small' for more characters per line
\usepackage{framed}
\definecolor{shadecolor}{RGB}{248,248,248}
\newenvironment{Shaded}{\begin{snugshade}}{\end{snugshade}}
\newcommand{\KeywordTok}[1]{\textcolor[rgb]{0.13,0.29,0.53}{\textbf{{#1}}}}
\newcommand{\DataTypeTok}[1]{\textcolor[rgb]{0.13,0.29,0.53}{{#1}}}
\newcommand{\DecValTok}[1]{\textcolor[rgb]{0.00,0.00,0.81}{{#1}}}
\newcommand{\BaseNTok}[1]{\textcolor[rgb]{0.00,0.00,0.81}{{#1}}}
\newcommand{\FloatTok}[1]{\textcolor[rgb]{0.00,0.00,0.81}{{#1}}}
\newcommand{\ConstantTok}[1]{\textcolor[rgb]{0.00,0.00,0.00}{{#1}}}
\newcommand{\CharTok}[1]{\textcolor[rgb]{0.31,0.60,0.02}{{#1}}}
\newcommand{\SpecialCharTok}[1]{\textcolor[rgb]{0.00,0.00,0.00}{{#1}}}
\newcommand{\StringTok}[1]{\textcolor[rgb]{0.31,0.60,0.02}{{#1}}}
\newcommand{\VerbatimStringTok}[1]{\textcolor[rgb]{0.31,0.60,0.02}{{#1}}}
\newcommand{\SpecialStringTok}[1]{\textcolor[rgb]{0.31,0.60,0.02}{{#1}}}
\newcommand{\ImportTok}[1]{{#1}}
\newcommand{\CommentTok}[1]{\textcolor[rgb]{0.56,0.35,0.01}{\textit{{#1}}}}
\newcommand{\DocumentationTok}[1]{\textcolor[rgb]{0.56,0.35,0.01}{\textbf{\textit{{#1}}}}}
\newcommand{\AnnotationTok}[1]{\textcolor[rgb]{0.56,0.35,0.01}{\textbf{\textit{{#1}}}}}
\newcommand{\CommentVarTok}[1]{\textcolor[rgb]{0.56,0.35,0.01}{\textbf{\textit{{#1}}}}}
\newcommand{\OtherTok}[1]{\textcolor[rgb]{0.56,0.35,0.01}{{#1}}}
\newcommand{\FunctionTok}[1]{\textcolor[rgb]{0.00,0.00,0.00}{{#1}}}
\newcommand{\VariableTok}[1]{\textcolor[rgb]{0.00,0.00,0.00}{{#1}}}
\newcommand{\ControlFlowTok}[1]{\textcolor[rgb]{0.13,0.29,0.53}{\textbf{{#1}}}}
\newcommand{\OperatorTok}[1]{\textcolor[rgb]{0.81,0.36,0.00}{\textbf{{#1}}}}
\newcommand{\BuiltInTok}[1]{{#1}}
\newcommand{\ExtensionTok}[1]{{#1}}
\newcommand{\PreprocessorTok}[1]{\textcolor[rgb]{0.56,0.35,0.01}{\textit{{#1}}}}
\newcommand{\AttributeTok}[1]{\textcolor[rgb]{0.77,0.63,0.00}{{#1}}}
\newcommand{\RegionMarkerTok}[1]{{#1}}
\newcommand{\InformationTok}[1]{\textcolor[rgb]{0.56,0.35,0.01}{\textbf{\textit{{#1}}}}}
\newcommand{\WarningTok}[1]{\textcolor[rgb]{0.56,0.35,0.01}{\textbf{\textit{{#1}}}}}
\newcommand{\AlertTok}[1]{\textcolor[rgb]{0.94,0.16,0.16}{{#1}}}
\newcommand{\ErrorTok}[1]{\textcolor[rgb]{0.64,0.00,0.00}{\textbf{{#1}}}}
\newcommand{\NormalTok}[1]{{#1}}
\usepackage{graphicx,grffile}
\makeatletter
\def\maxwidth{\ifdim\Gin@nat@width>\linewidth\linewidth\else\Gin@nat@width\fi}
\def\maxheight{\ifdim\Gin@nat@height>\textheight\textheight\else\Gin@nat@height\fi}
\makeatother
% Scale images if necessary, so that they will not overflow the page
% margins by default, and it is still possible to overwrite the defaults
% using explicit options in \includegraphics[width, height, ...]{}
\setkeys{Gin}{width=\maxwidth,height=\maxheight,keepaspectratio}
\setlength{\parindent}{0pt}
\setlength{\parskip}{6pt plus 2pt minus 1pt}
\setlength{\emergencystretch}{3em}  % prevent overfull lines
\providecommand{\tightlist}{%
  \setlength{\itemsep}{0pt}\setlength{\parskip}{0pt}}
\setcounter{secnumdepth}{0}

%%% Use protect on footnotes to avoid problems with footnotes in titles
\let\rmarkdownfootnote\footnote%
\def\footnote{\protect\rmarkdownfootnote}

%%% Change title format to be more compact
\usepackage{titling}

% Create subtitle command for use in maketitle
\newcommand{\subtitle}[1]{
  \posttitle{
    \begin{center}\large#1\end{center}
    }
}

\setlength{\droptitle}{-2em}
  \title{Homework 2}
  \pretitle{\vspace{\droptitle}\centering\huge}
  \posttitle{\par}
  \author{Jessica Temporal 7547611}
  \preauthor{\centering\large\emph}
  \postauthor{\par}
  \predate{\centering\large\emph}
  \postdate{\par}
  \date{November 10, 2016}



% Redefines (sub)paragraphs to behave more like sections
\ifx\paragraph\undefined\else
\let\oldparagraph\paragraph
\renewcommand{\paragraph}[1]{\oldparagraph{#1}\mbox{}}
\fi
\ifx\subparagraph\undefined\else
\let\oldsubparagraph\subparagraph
\renewcommand{\subparagraph}[1]{\oldsubparagraph{#1}\mbox{}}
\fi

\begin{document}
\maketitle

{
\setcounter{tocdepth}{3}
\tableofcontents
}
\newpage

\subsection{1. Read the provided text file into R, making sure to
account for headers. This is a truncated cDNA array expression file that
has only the Cy5 and Cy3 foreground and background values for 5 separate
arrays. The numbers in the suffix of the names are time points (though
not necessary for our purposes). (2.5
pts)}\label{read-the-provided-text-file-into-r-making-sure-to-account-for-headers.-this-is-a-truncated-cdna-array-expression-file-that-has-only-the-cy5-and-cy3-foreground-and-background-values-for-5-separate-arrays.-the-numbers-in-the-suffix-of-the-names-are-time-points-though-not-necessary-for-our-purposes.-2.5-pts}

\begin{Shaded}
\begin{Highlighting}[]
\NormalTok{file <-}\StringTok{ 'five.txt'}
\NormalTok{data <-}\StringTok{ }\KeywordTok{read.table}\NormalTok{(file, }\DataTypeTok{header =} \NormalTok{T)}
\end{Highlighting}
\end{Shaded}

\subsection{2. Load the sma library and the MouseArray data file (with
data()). (2.5
pts)}\label{load-the-sma-library-and-the-mousearray-data-file-with-data.-2.5-pts}

\begin{Shaded}
\begin{Highlighting}[]
\KeywordTok{library}\NormalTok{(sma)}
\KeywordTok{data}\NormalTok{(MouseArray)}
\end{Highlighting}
\end{Shaded}

\subsection{3. Using list commands, turn the five\_arrays matrix file
into the list object that is similar to the mouse.data example. See
hints below. (10
pts)}\label{using-list-commands-turn-the-fiveux5farrays-matrix-file-into-the-list-object-that-is-similar-to-the-mouse.data-example.-see-hints-below.-10-pts}

\begin{Shaded}
\begin{Highlighting}[]
\NormalTok{five.data <-}\StringTok{ }\KeywordTok{list}\NormalTok{(}\StringTok{'R'} \NormalTok{=}\StringTok{ }\NormalTok{data[,}\KeywordTok{c}\NormalTok{(}\DecValTok{1}\NormalTok{:}\DecValTok{5}\NormalTok{)],}
                  \StringTok{'G'} \NormalTok{=}\StringTok{ }\NormalTok{data[, }\KeywordTok{c}\NormalTok{(}\DecValTok{6}\NormalTok{:}\DecValTok{10}\NormalTok{)],}
                  \StringTok{'Rb'} \NormalTok{=}\StringTok{ }\NormalTok{data[,}\KeywordTok{c}\NormalTok{(}\DecValTok{11}\NormalTok{:}\DecValTok{15}\NormalTok{)],}
                  \StringTok{'Gb'} \NormalTok{=}\StringTok{ }\NormalTok{data[, }\KeywordTok{c}\NormalTok{(}\DecValTok{16}\NormalTok{:}\DecValTok{20}\NormalTok{)]}
                  \NormalTok{)}
\end{Highlighting}
\end{Shaded}

\newpage

\subsection{4. Now create a plot of both un-normalized and lowess
normalized data for chip \#5 (image.id argument) using the plot.mva
function. Make sure to use a par command to put both plots on the same
page. Title both plots. For the layout argument in plot.mva(), use the
mouse.setup file. (15
pts)}\label{now-create-a-plot-of-both-un-normalized-and-lowess-normalized-data-for-chip-5-image.id-argument-using-the-plot.mva-function.-make-sure-to-use-a-par-command-to-put-both-plots-on-the-same-page.-title-both-plots.-for-the-layout-argument-in-plot.mva-use-the-mouse.setup-file.-15-pts}

\begin{Shaded}
\begin{Highlighting}[]
\KeywordTok{par}\NormalTok{(}\DataTypeTok{mfrow=}\KeywordTok{c}\NormalTok{(}\DecValTok{2}\NormalTok{,}\DecValTok{1}\NormalTok{))}
\KeywordTok{plot.mva}\NormalTok{(five.data, }\DataTypeTok{layout =} \NormalTok{mouse.setup, }\DataTypeTok{plot.type =} \StringTok{'r'}\NormalTok{,}
         \DataTypeTok{norm =} \StringTok{"n"}\NormalTok{, }\DataTypeTok{extra.type =} \StringTok{"pci"}\NormalTok{, }\DataTypeTok{image.id =} \DecValTok{5}\NormalTok{, }\DataTypeTok{main =} \StringTok{"non-normalized - Chip #5"}\NormalTok{)}
\KeywordTok{plot.mva}\NormalTok{(five.data, }\DataTypeTok{layout =} \NormalTok{mouse.setup, }\DataTypeTok{plot.type =} \StringTok{'n'}\NormalTok{,}
         \DataTypeTok{norm =} \StringTok{"l"}\NormalTok{, }\DataTypeTok{extra.type =} \StringTok{"pci"}\NormalTok{, }\DataTypeTok{image.id =} \DecValTok{5}\NormalTok{, }\DataTypeTok{main =} \StringTok{"lowess normalized - Chip #5"}\NormalTok{)}
\end{Highlighting}
\end{Shaded}

\includegraphics{homework2_files/figure-latex/unnamed-chunk-4-1.pdf}

\subsection{5. How do the 2 plots differ? What has lowess normalization
done to the distribution of expression values? (5
pts)}\label{how-do-the-2-plots-differ-what-has-lowess-normalization-done-to-the-distribution-of-expression-values-5-pts}

A normalização fez com que os valores de expressão ficassem mais
uniformmente distribuidos em torno do eixo traçado na linha zero.
Enquanto os dados não normalizados tem uma janela maior de variação,
alterando a linha desenahada que gira perto do -5.

\end{document}
