\documentclass[]{article}
\usepackage{lmodern}
\usepackage{amssymb,amsmath}
\usepackage{ifxetex,ifluatex}
\usepackage{fixltx2e} % provides \textsubscript
\ifnum 0\ifxetex 1\fi\ifluatex 1\fi=0 % if pdftex
  \usepackage[T1]{fontenc}
  \usepackage[utf8]{inputenc}
\else % if luatex or xelatex
  \ifxetex
    \usepackage{mathspec}
  \else
    \usepackage{fontspec}
  \fi
  \defaultfontfeatures{Ligatures=TeX,Scale=MatchLowercase}
  \newcommand{\euro}{€}
\fi
% use upquote if available, for straight quotes in verbatim environments
\IfFileExists{upquote.sty}{\usepackage{upquote}}{}
% use microtype if available
\IfFileExists{microtype.sty}{%
\usepackage{microtype}
\UseMicrotypeSet[protrusion]{basicmath} % disable protrusion for tt fonts
}{}
\usepackage[margin=1in]{geometry}
\usepackage{hyperref}
\PassOptionsToPackage{usenames,dvipsnames}{color} % color is loaded by hyperref
\hypersetup{unicode=true,
            pdftitle={Homework 1},
            pdfauthor={Jessica Temporal 7547611},
            pdfborder={0 0 0},
            breaklinks=true}
\urlstyle{same}  % don't use monospace font for urls
\usepackage{color}
\usepackage{fancyvrb}
\newcommand{\VerbBar}{|}
\newcommand{\VERB}{\Verb[commandchars=\\\{\}]}
\DefineVerbatimEnvironment{Highlighting}{Verbatim}{commandchars=\\\{\}}
% Add ',fontsize=\small' for more characters per line
\usepackage{framed}
\definecolor{shadecolor}{RGB}{248,248,248}
\newenvironment{Shaded}{\begin{snugshade}}{\end{snugshade}}
\newcommand{\KeywordTok}[1]{\textcolor[rgb]{0.13,0.29,0.53}{\textbf{{#1}}}}
\newcommand{\DataTypeTok}[1]{\textcolor[rgb]{0.13,0.29,0.53}{{#1}}}
\newcommand{\DecValTok}[1]{\textcolor[rgb]{0.00,0.00,0.81}{{#1}}}
\newcommand{\BaseNTok}[1]{\textcolor[rgb]{0.00,0.00,0.81}{{#1}}}
\newcommand{\FloatTok}[1]{\textcolor[rgb]{0.00,0.00,0.81}{{#1}}}
\newcommand{\ConstantTok}[1]{\textcolor[rgb]{0.00,0.00,0.00}{{#1}}}
\newcommand{\CharTok}[1]{\textcolor[rgb]{0.31,0.60,0.02}{{#1}}}
\newcommand{\SpecialCharTok}[1]{\textcolor[rgb]{0.00,0.00,0.00}{{#1}}}
\newcommand{\StringTok}[1]{\textcolor[rgb]{0.31,0.60,0.02}{{#1}}}
\newcommand{\VerbatimStringTok}[1]{\textcolor[rgb]{0.31,0.60,0.02}{{#1}}}
\newcommand{\SpecialStringTok}[1]{\textcolor[rgb]{0.31,0.60,0.02}{{#1}}}
\newcommand{\ImportTok}[1]{{#1}}
\newcommand{\CommentTok}[1]{\textcolor[rgb]{0.56,0.35,0.01}{\textit{{#1}}}}
\newcommand{\DocumentationTok}[1]{\textcolor[rgb]{0.56,0.35,0.01}{\textbf{\textit{{#1}}}}}
\newcommand{\AnnotationTok}[1]{\textcolor[rgb]{0.56,0.35,0.01}{\textbf{\textit{{#1}}}}}
\newcommand{\CommentVarTok}[1]{\textcolor[rgb]{0.56,0.35,0.01}{\textbf{\textit{{#1}}}}}
\newcommand{\OtherTok}[1]{\textcolor[rgb]{0.56,0.35,0.01}{{#1}}}
\newcommand{\FunctionTok}[1]{\textcolor[rgb]{0.00,0.00,0.00}{{#1}}}
\newcommand{\VariableTok}[1]{\textcolor[rgb]{0.00,0.00,0.00}{{#1}}}
\newcommand{\ControlFlowTok}[1]{\textcolor[rgb]{0.13,0.29,0.53}{\textbf{{#1}}}}
\newcommand{\OperatorTok}[1]{\textcolor[rgb]{0.81,0.36,0.00}{\textbf{{#1}}}}
\newcommand{\BuiltInTok}[1]{{#1}}
\newcommand{\ExtensionTok}[1]{{#1}}
\newcommand{\PreprocessorTok}[1]{\textcolor[rgb]{0.56,0.35,0.01}{\textit{{#1}}}}
\newcommand{\AttributeTok}[1]{\textcolor[rgb]{0.77,0.63,0.00}{{#1}}}
\newcommand{\RegionMarkerTok}[1]{{#1}}
\newcommand{\InformationTok}[1]{\textcolor[rgb]{0.56,0.35,0.01}{\textbf{\textit{{#1}}}}}
\newcommand{\WarningTok}[1]{\textcolor[rgb]{0.56,0.35,0.01}{\textbf{\textit{{#1}}}}}
\newcommand{\AlertTok}[1]{\textcolor[rgb]{0.94,0.16,0.16}{{#1}}}
\newcommand{\ErrorTok}[1]{\textcolor[rgb]{0.64,0.00,0.00}{\textbf{{#1}}}}
\newcommand{\NormalTok}[1]{{#1}}
\usepackage{graphicx,grffile}
\makeatletter
\def\maxwidth{\ifdim\Gin@nat@width>\linewidth\linewidth\else\Gin@nat@width\fi}
\def\maxheight{\ifdim\Gin@nat@height>\textheight\textheight\else\Gin@nat@height\fi}
\makeatother
% Scale images if necessary, so that they will not overflow the page
% margins by default, and it is still possible to overwrite the defaults
% using explicit options in \includegraphics[width, height, ...]{}
\setkeys{Gin}{width=\maxwidth,height=\maxheight,keepaspectratio}
\setlength{\parindent}{0pt}
\setlength{\parskip}{6pt plus 2pt minus 1pt}
\setlength{\emergencystretch}{3em}  % prevent overfull lines
\providecommand{\tightlist}{%
  \setlength{\itemsep}{0pt}\setlength{\parskip}{0pt}}
\setcounter{secnumdepth}{0}

%%% Use protect on footnotes to avoid problems with footnotes in titles
\let\rmarkdownfootnote\footnote%
\def\footnote{\protect\rmarkdownfootnote}

%%% Change title format to be more compact
\usepackage{titling}

% Create subtitle command for use in maketitle
\newcommand{\subtitle}[1]{
  \posttitle{
    \begin{center}\large#1\end{center}
    }
}

\setlength{\droptitle}{-2em}
  \title{Homework 1}
  \pretitle{\vspace{\droptitle}\centering\huge}
  \posttitle{\par}
  \author{Jessica Temporal 7547611}
  \preauthor{\centering\large\emph}
  \postauthor{\par}
  \predate{\centering\large\emph}
  \postdate{\par}
  \date{August 15, 2016}



% Redefines (sub)paragraphs to behave more like sections
\ifx\paragraph\undefined\else
\let\oldparagraph\paragraph
\renewcommand{\paragraph}[1]{\oldparagraph{#1}\mbox{}}
\fi
\ifx\subparagraph\undefined\else
\let\oldsubparagraph\subparagraph
\renewcommand{\subparagraph}[1]{\oldsubparagraph{#1}\mbox{}}
\fi

\begin{document}
\maketitle

{
\setcounter{tocdepth}{3}
\tableofcontents
}
\newpage

\subsection{1) Load the Golub ALL/AML training set data file into R. Set
the row names to the first column values (Affymetrix fragment names) and
remove the first column. Look at the dimensions and verify that you have
38 arrays and 7,129
genes.}\label{load-the-golub-allaml-training-set-data-file-into-r.-set-the-row-names-to-the-first-column-values-affymetrix-fragment-names-and-remove-the-first-column.-look-at-the-dimensions-and-verify-that-you-have-38-arrays-and-7129-genes.}

\begin{Shaded}
\begin{Highlighting}[]
\NormalTok{file <-}\StringTok{ "golubTrain.txt"}
\NormalTok{golub_data <-}\StringTok{ }\KeywordTok{read.table}\NormalTok{(file, }\DataTypeTok{header =} \NormalTok{T)}
\KeywordTok{rownames}\NormalTok{(golub_data) <-}\StringTok{ }\KeywordTok{as.character}\NormalTok{(golub_data[,}\DecValTok{1}\NormalTok{])}
\NormalTok{golub_data$Gene <-}\StringTok{ }\OtherTok{NULL}
\KeywordTok{dim}\NormalTok{(golub_data)}
\end{Highlighting}
\end{Shaded}

\begin{verbatim}
## [1] 7129   38
\end{verbatim}

\subsection{2) Download and load the 3 class annotation file
(golubTrainClass2.txt) for this data set into
R.}\label{download-and-load-the-3-class-annotation-file-golubtrainclass2.txt-for-this-data-set-into-r.}

\begin{Shaded}
\begin{Highlighting}[]
\NormalTok{file2 <-}\StringTok{ "golubTrainClass2.txt"}
\NormalTok{class_data <-}\StringTok{ }\KeywordTok{read.table}\NormalTok{(file2)}
\end{Highlighting}
\end{Shaded}

\subsection{3) Cast the data to a data
frame.}\label{cast-the-data-to-a-data-frame.}

\begin{Shaded}
\begin{Highlighting}[]
\NormalTok{golub_df <-}\StringTok{ }\KeywordTok{as.data.frame}\NormalTok{(golub_data)}
\end{Highlighting}
\end{Shaded}

\subsection{4) Make the names of the data frame the annotation file
labels (remember that depending on how you read in the annotation file,
it may be a dataframe with 1
column.}\label{make-the-names-of-the-data-frame-the-annotation-file-labels-remember-that-depending-on-how-you-read-in-the-annotation-file-it-may-be-a-dataframe-with-1-column.}

\begin{Shaded}
\begin{Highlighting}[]
\KeywordTok{names}\NormalTok{(golub_df) <-}\StringTok{ }\NormalTok{class_data[,}\DecValTok{1}\NormalTok{]}
\end{Highlighting}
\end{Shaded}

\subsection{5) Subset the data by the first 100
genes.}\label{subset-the-data-by-the-first-100-genes.}

\begin{Shaded}
\begin{Highlighting}[]
\NormalTok{golub_sub <-}\StringTok{ }\NormalTok{golub_df[}\DecValTok{1}\NormalTok{:}\DecValTok{100}\NormalTok{,]}
\end{Highlighting}
\end{Shaded}

\newpage

\subsection{6) When utilizing only the first 100 genes, there exists one
aberrant outlier sample. Identify this outlier sample using the
following visual
plots:}\label{when-utilizing-only-the-first-100-genes-there-exists-one-aberrant-outlier-sample.-identify-this-outlier-sample-using-the-following-visual-plots}

\subsubsection{6.1) Correlation plot (heat
map)}\label{correlation-plot-heat-map}

\begin{Shaded}
\begin{Highlighting}[]
\NormalTok{golub_cor <-}\StringTok{ }\KeywordTok{cor}\NormalTok{(golub_sub)}
\KeywordTok{image}\NormalTok{(golub_cor, }\DataTypeTok{main =} \StringTok{"Correlation plot - Golub Data"}\NormalTok{)}
\end{Highlighting}
\end{Shaded}

\includegraphics{homework1_files/figure-latex/unnamed-chunk-6-1.pdf}
\newpage

\subsubsection{6.2) Hierarchical clustering
dendrogram}\label{hierarchical-clustering-dendrogram}

\begin{Shaded}
\begin{Highlighting}[]
\NormalTok{golub_dist <-}\StringTok{ }\KeywordTok{dist}\NormalTok{(}\KeywordTok{t}\NormalTok{(golub_sub))}
\NormalTok{golub_hclust <-}\StringTok{ }\KeywordTok{hclust}\NormalTok{(golub_dist)}
\KeywordTok{plot}\NormalTok{(golub_hclust, }\DataTypeTok{main =} \StringTok{"Hierarchical dendrogram - Golub Data"}\NormalTok{, }\DataTypeTok{labels =} \KeywordTok{c}\NormalTok{(}\DecValTok{1}\NormalTok{:}\DecValTok{38}\NormalTok{))}
\end{Highlighting}
\end{Shaded}

\includegraphics{homework1_files/figure-latex/unnamed-chunk-7-1.pdf}
\newpage

\subsubsection{6.3) CV vs.~mean plot}\label{cv-vs.mean-plot}

\begin{Shaded}
\begin{Highlighting}[]
\NormalTok{golub_sub_mean <-}\StringTok{ }\KeywordTok{apply}\NormalTok{(golub_sub, }\DecValTok{2}\NormalTok{, mean)}
\NormalTok{golub_sub_cv <-}\StringTok{ }\NormalTok{(}\KeywordTok{apply}\NormalTok{(golub_sub, }\DecValTok{2}\NormalTok{, sd))/golub_sub_mean}
\KeywordTok{plot}\NormalTok{(golub_sub_cv, golub_sub_mean, }\DataTypeTok{xlab =} \StringTok{"CV"}\NormalTok{, }\DataTypeTok{ylab =} \StringTok{"Mean"}\NormalTok{,}
     \DataTypeTok{main =} \StringTok{"Golub - CV vs. mean"}\NormalTok{)}
\KeywordTok{text}\NormalTok{(golub_sub_cv, golub_sub_mean, }\DataTypeTok{labels =} \KeywordTok{c}\NormalTok{(}\DecValTok{1}\NormalTok{:}\KeywordTok{ncol}\NormalTok{(golub_sub)), }\DataTypeTok{pos =} \DecValTok{3}\NormalTok{)}
\end{Highlighting}
\end{Shaded}

\includegraphics{homework1_files/figure-latex/unnamed-chunk-8-1.pdf}

\subsection{7) Now download and load the Spellman yeast data set.
Remember to set the row names to the first column as you did before with
the leukemia
dataset.}\label{now-download-and-load-the-spellman-yeast-data-set.-remember-to-set-the-row-names-to-the-first-column-as-you-did-before-with-the-leukemia-dataset.}

\begin{Shaded}
\begin{Highlighting}[]
\NormalTok{spell <-}\StringTok{ "spellman.txt"}
\NormalTok{spell_data <-}\StringTok{ }\KeywordTok{read.table}\NormalTok{(spell, }\DataTypeTok{header =} \NormalTok{T)}
\KeywordTok{rownames}\NormalTok{(spell_data) <-}\StringTok{ }\KeywordTok{as.character}\NormalTok{(spell_data$row.names)}
\NormalTok{spell_data$row.names <-}\StringTok{ }\OtherTok{NULL}
\end{Highlighting}
\end{Shaded}

\subsection{8) Cast the data to a data frame and subset to only work
with the cdc28 experiment
samples.}\label{cast-the-data-to-a-data-frame-and-subset-to-only-work-with-the-cdc28-experiment-samples.}

\begin{Shaded}
\begin{Highlighting}[]
\NormalTok{spell_df <-}\StringTok{ }\KeywordTok{as.data.frame}\NormalTok{(spell_data)}
\NormalTok{cdc28 <-}\StringTok{ }\KeywordTok{subset}\NormalTok{(spell_df, }\DataTypeTok{select =} \KeywordTok{grep}\NormalTok{(}\StringTok{"cdc28"}\NormalTok{, }\KeywordTok{names}\NormalTok{(spell_df)))}
\end{Highlighting}
\end{Shaded}

\subsection{\texorpdfstring{9) Use both the function and call to the
function below to fill all ``NA'' values with the computed row
means.}{9) Use both the function and call to the function below to fill all NA values with the computed row means.}}\label{use-both-the-function-and-call-to-the-function-below-to-fill-all-na-values-with-the-computed-row-means.}

\begin{Shaded}
\begin{Highlighting}[]
\NormalTok{miss.fill <-}\StringTok{ }\NormalTok{function(x) \{}
    \NormalTok{if(}\KeywordTok{sum}\NormalTok{(}\KeywordTok{is.na}\NormalTok{(}\KeywordTok{as.numeric}\NormalTok{(x))) ==}\DecValTok{17} \NormalTok{) \{}
        \NormalTok{x[}\KeywordTok{is.na}\NormalTok{(x)] <-}\DecValTok{0}
    \NormalTok{\}}
    \NormalTok{if(}\KeywordTok{sum}\NormalTok{(}\KeywordTok{is.na}\NormalTok{(}\KeywordTok{as.numeric}\NormalTok{(x))) >}\StringTok{ }\DecValTok{0} \NormalTok{&}\StringTok{ }\KeywordTok{sum}\NormalTok{(}\KeywordTok{is.na}\NormalTok{(}\KeywordTok{as.numeric}\NormalTok{(x))) <}\DecValTok{17} \NormalTok{) \{}
        \NormalTok{x[}\KeywordTok{is.na}\NormalTok{(x)] <-}\StringTok{ }\KeywordTok{mean}\NormalTok{(}\KeywordTok{as.numeric}\NormalTok{(x[!}\KeywordTok{is.na}\NormalTok{(x)]))}
    \NormalTok{\}}
    \KeywordTok{return}\NormalTok{(x)}
\NormalTok{\}}

\NormalTok{cdc28_fill <-}\StringTok{ }\KeywordTok{as.data.frame}\NormalTok{(}\KeywordTok{t}\NormalTok{(}\KeywordTok{apply}\NormalTok{(cdc28, }\DecValTok{1}\NormalTok{, miss.fill)))}
\NormalTok{cdc28_fill <-}\StringTok{ }\KeywordTok{round}\NormalTok{(cdc28_fill, }\DecValTok{2}\NormalTok{)}
\end{Highlighting}
\end{Shaded}

\subsection{10) Calculate the kmeans clustering method on all 6,178
genes, using 10 cluster centers and 100
iterations.}\label{calculate-the-kmeans-clustering-method-on-all-6178-genes-using-10-cluster-centers-and-100-iterations.}

\begin{Shaded}
\begin{Highlighting}[]
\NormalTok{cdc28_kmeans <-}\StringTok{ }\KeywordTok{kmeans}\NormalTok{(cdc28_fill, }\DataTypeTok{centers =} \DecValTok{10}\NormalTok{, }\DataTypeTok{iter.max =} \DecValTok{100}\NormalTok{)}
\end{Highlighting}
\end{Shaded}

\subsection{\texorpdfstring{11) Look for gene \#2 (YAL002W) and find the
cluster that it belongs to. Using these genes, calculate the distance
from each gene to gene \#2 (use manhattan distance in \texttt{dist()}
function)}{11) Look for gene \#2 (YAL002W) and find the cluster that it belongs to. Using these genes, calculate the distance from each gene to gene \#2 (use manhattan distance in dist() function)}}\label{look-for-gene-2-yal002w-and-find-the-cluster-that-it-belongs-to.-using-these-genes-calculate-the-distance-from-each-gene-to-gene-2-use-manhattan-distance-in-dist-function}

\begin{Shaded}
\begin{Highlighting}[]
\NormalTok{groups <-}\StringTok{ }\NormalTok{cdc28_kmeans$cluster}
\KeywordTok{head}\NormalTok{(groups)}
\end{Highlighting}
\end{Shaded}

\begin{verbatim}
## YAL001C YAL002W YAL003W YAL004W YAL005C YAL007C 
##       4       4       3       2       2       6
\end{verbatim}

\begin{Shaded}
\begin{Highlighting}[]
\NormalTok{groups[}\StringTok{'YAL002W'}\NormalTok{] }\CommentTok{# cluster 4}
\end{Highlighting}
\end{Shaded}

\begin{verbatim}
## YAL002W 
##       4
\end{verbatim}

\begin{Shaded}
\begin{Highlighting}[]
\NormalTok{cluster <-}\StringTok{ }\NormalTok{groups ==}\StringTok{ }\DecValTok{4}
\NormalTok{cluster <-}\StringTok{ }\KeywordTok{dimnames}\NormalTok{(cdc28_fill)[[}\DecValTok{1}\NormalTok{]][cluster]}
\NormalTok{dist2genes <-}\StringTok{ }\KeywordTok{dist}\NormalTok{(cdc28_fill[cluster,], }\DataTypeTok{method =} \StringTok{'man'}\NormalTok{)}
\end{Highlighting}
\end{Shaded}

\subsection{\texorpdfstring{12) Cast the distance object to a matrix and
get the column that gene \#2 corresponds to. Hint: If gene \#2 is in
column \#1 of your matrix, get the corresponding cluster member
distances with:
\texttt{gene.dist\ \textless{}-\ gene.dist{[}2:nrow(gene.dist),1{]}} If
gene \#2 is in column \#4 of your matrix, get the corresponding cluster
member distances with:
\texttt{gene.dist\ \textless{}-\ gene.dist{[}c(1:3,5:nrow(gene.dist)),4{]}}}{12) Cast the distance object to a matrix and get the column that gene \#2 corresponds to. Hint: If gene \#2 is in column \#1 of your matrix, get the corresponding cluster member distances with: gene.dist \textless{}- gene.dist{[}2:nrow(gene.dist),1{]} If gene \#2 is in column \#4 of your matrix, get the corresponding cluster member distances with: gene.dist \textless{}- gene.dist{[}c(1:3,5:nrow(gene.dist)),4{]}}}\label{cast-the-distance-object-to-a-matrix-and-get-the-column-that-gene-2-corresponds-to.-hint-if-gene-2-is-in-column-1-of-your-matrix-get-the-corresponding-cluster-member-distances-with-gene.dist---gene.dist2nrowgene.dist1-if-gene-2-is-in-column-4-of-your-matrix-get-the-corresponding-cluster-member-distances-with-gene.dist---gene.distc135nrowgene.dist4}

\begin{Shaded}
\begin{Highlighting}[]
\NormalTok{dist_mat <-}\StringTok{ }\KeywordTok{as.matrix}\NormalTok{(dist2genes)}
\NormalTok{gene2 <-}\StringTok{ }\NormalTok{dist_mat[}\DecValTok{2}\NormalTok{:}\KeywordTok{nrow}\NormalTok{(dist_mat),}\StringTok{'YAL002W'}\NormalTok{]}
\end{Highlighting}
\end{Shaded}

\subsection{13) Get the weights of each gene as a percentage of the sum
of distance values. Assuming that the first array (cdc28\_0) is missing
a value for gene \#2, calculate the weighted mean from the gene weight
vector for this missing value. Print out this weighted mean
value.}\label{get-the-weights-of-each-gene-as-a-percentage-of-the-sum-of-distance-values.-assuming-that-the-first-array-cdc28ux5f0-is-missing-a-value-for-gene-2-calculate-the-weighted-mean-from-the-gene-weight-vector-for-this-missing-value.-print-out-this-weighted-mean-value.}

\begin{Shaded}
\begin{Highlighting}[]
\NormalTok{weights <-}\StringTok{ }\KeywordTok{as.numeric}\NormalTok{(gene2/}\KeywordTok{sum}\NormalTok{(gene2))}
\NormalTok{weighted_mean <-}\StringTok{ }\KeywordTok{weighted.mean}\NormalTok{(}\DataTypeTok{x =} \KeywordTok{as.numeric}\NormalTok{(gene2), }\DataTypeTok{w =} \NormalTok{weights)}
\NormalTok{weighted_mean}
\end{Highlighting}
\end{Shaded}

\begin{verbatim}
## [1] 5.527842
\end{verbatim}

\end{document}
