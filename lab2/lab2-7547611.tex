\documentclass[]{article}
\usepackage{lmodern}
\usepackage{amssymb,amsmath}
\usepackage{ifxetex,ifluatex}
\usepackage{fixltx2e} % provides \textsubscript
\ifnum 0\ifxetex 1\fi\ifluatex 1\fi=0 % if pdftex
  \usepackage[T1]{fontenc}
  \usepackage[utf8]{inputenc}
\else % if luatex or xelatex
  \ifxetex
    \usepackage{mathspec}
  \else
    \usepackage{fontspec}
  \fi
  \defaultfontfeatures{Ligatures=TeX,Scale=MatchLowercase}
  \newcommand{\euro}{€}
\fi
% use upquote if available, for straight quotes in verbatim environments
\IfFileExists{upquote.sty}{\usepackage{upquote}}{}
% use microtype if available
\IfFileExists{microtype.sty}{%
\usepackage{microtype}
\UseMicrotypeSet[protrusion]{basicmath} % disable protrusion for tt fonts
}{}
\usepackage[margin=1in]{geometry}
\usepackage{hyperref}
\PassOptionsToPackage{usenames,dvipsnames}{color} % color is loaded by hyperref
\hypersetup{unicode=true,
            pdftitle={Lab2},
            pdfauthor={Jessica Temporal 7547611},
            pdfsubject={Data visualization},
            pdfborder={0 0 0},
            breaklinks=true}
\urlstyle{same}  % don't use monospace font for urls
\usepackage{color}
\usepackage{fancyvrb}
\newcommand{\VerbBar}{|}
\newcommand{\VERB}{\Verb[commandchars=\\\{\}]}
\DefineVerbatimEnvironment{Highlighting}{Verbatim}{commandchars=\\\{\}}
% Add ',fontsize=\small' for more characters per line
\usepackage{framed}
\definecolor{shadecolor}{RGB}{248,248,248}
\newenvironment{Shaded}{\begin{snugshade}}{\end{snugshade}}
\newcommand{\KeywordTok}[1]{\textcolor[rgb]{0.13,0.29,0.53}{\textbf{{#1}}}}
\newcommand{\DataTypeTok}[1]{\textcolor[rgb]{0.13,0.29,0.53}{{#1}}}
\newcommand{\DecValTok}[1]{\textcolor[rgb]{0.00,0.00,0.81}{{#1}}}
\newcommand{\BaseNTok}[1]{\textcolor[rgb]{0.00,0.00,0.81}{{#1}}}
\newcommand{\FloatTok}[1]{\textcolor[rgb]{0.00,0.00,0.81}{{#1}}}
\newcommand{\ConstantTok}[1]{\textcolor[rgb]{0.00,0.00,0.00}{{#1}}}
\newcommand{\CharTok}[1]{\textcolor[rgb]{0.31,0.60,0.02}{{#1}}}
\newcommand{\SpecialCharTok}[1]{\textcolor[rgb]{0.00,0.00,0.00}{{#1}}}
\newcommand{\StringTok}[1]{\textcolor[rgb]{0.31,0.60,0.02}{{#1}}}
\newcommand{\VerbatimStringTok}[1]{\textcolor[rgb]{0.31,0.60,0.02}{{#1}}}
\newcommand{\SpecialStringTok}[1]{\textcolor[rgb]{0.31,0.60,0.02}{{#1}}}
\newcommand{\ImportTok}[1]{{#1}}
\newcommand{\CommentTok}[1]{\textcolor[rgb]{0.56,0.35,0.01}{\textit{{#1}}}}
\newcommand{\DocumentationTok}[1]{\textcolor[rgb]{0.56,0.35,0.01}{\textbf{\textit{{#1}}}}}
\newcommand{\AnnotationTok}[1]{\textcolor[rgb]{0.56,0.35,0.01}{\textbf{\textit{{#1}}}}}
\newcommand{\CommentVarTok}[1]{\textcolor[rgb]{0.56,0.35,0.01}{\textbf{\textit{{#1}}}}}
\newcommand{\OtherTok}[1]{\textcolor[rgb]{0.56,0.35,0.01}{{#1}}}
\newcommand{\FunctionTok}[1]{\textcolor[rgb]{0.00,0.00,0.00}{{#1}}}
\newcommand{\VariableTok}[1]{\textcolor[rgb]{0.00,0.00,0.00}{{#1}}}
\newcommand{\ControlFlowTok}[1]{\textcolor[rgb]{0.13,0.29,0.53}{\textbf{{#1}}}}
\newcommand{\OperatorTok}[1]{\textcolor[rgb]{0.81,0.36,0.00}{\textbf{{#1}}}}
\newcommand{\BuiltInTok}[1]{{#1}}
\newcommand{\ExtensionTok}[1]{{#1}}
\newcommand{\PreprocessorTok}[1]{\textcolor[rgb]{0.56,0.35,0.01}{\textit{{#1}}}}
\newcommand{\AttributeTok}[1]{\textcolor[rgb]{0.77,0.63,0.00}{{#1}}}
\newcommand{\RegionMarkerTok}[1]{{#1}}
\newcommand{\InformationTok}[1]{\textcolor[rgb]{0.56,0.35,0.01}{\textbf{\textit{{#1}}}}}
\newcommand{\WarningTok}[1]{\textcolor[rgb]{0.56,0.35,0.01}{\textbf{\textit{{#1}}}}}
\newcommand{\AlertTok}[1]{\textcolor[rgb]{0.94,0.16,0.16}{{#1}}}
\newcommand{\ErrorTok}[1]{\textcolor[rgb]{0.64,0.00,0.00}{\textbf{{#1}}}}
\newcommand{\NormalTok}[1]{{#1}}
\usepackage{graphicx,grffile}
\makeatletter
\def\maxwidth{\ifdim\Gin@nat@width>\linewidth\linewidth\else\Gin@nat@width\fi}
\def\maxheight{\ifdim\Gin@nat@height>\textheight\textheight\else\Gin@nat@height\fi}
\makeatother
% Scale images if necessary, so that they will not overflow the page
% margins by default, and it is still possible to overwrite the defaults
% using explicit options in \includegraphics[width, height, ...]{}
\setkeys{Gin}{width=\maxwidth,height=\maxheight,keepaspectratio}
\setlength{\parindent}{0pt}
\setlength{\parskip}{6pt plus 2pt minus 1pt}
\setlength{\emergencystretch}{3em}  % prevent overfull lines
\providecommand{\tightlist}{%
  \setlength{\itemsep}{0pt}\setlength{\parskip}{0pt}}
\setcounter{secnumdepth}{0}

%%% Use protect on footnotes to avoid problems with footnotes in titles
\let\rmarkdownfootnote\footnote%
\def\footnote{\protect\rmarkdownfootnote}

%%% Change title format to be more compact
\usepackage{titling}

% Create subtitle command for use in maketitle
\newcommand{\subtitle}[1]{
  \posttitle{
    \begin{center}\large#1\end{center}
    }
}

\setlength{\droptitle}{-2em}
  \title{Lab2}
  \pretitle{\vspace{\droptitle}\centering\huge}
  \posttitle{\par}
\subtitle{Data visualization}
  \author{Jessica Temporal 7547611}
  \preauthor{\centering\large\emph}
  \postauthor{\par}
  \predate{\centering\large\emph}
  \postdate{\par}
  \date{August 27, 2016}



% Redefines (sub)paragraphs to behave more like sections
\ifx\paragraph\undefined\else
\let\oldparagraph\paragraph
\renewcommand{\paragraph}[1]{\oldparagraph{#1}\mbox{}}
\fi
\ifx\subparagraph\undefined\else
\let\oldsubparagraph\subparagraph
\renewcommand{\subparagraph}[1]{\oldsubparagraph{#1}\mbox{}}
\fi

\begin{document}
\maketitle

{
\setcounter{tocdepth}{3}
\tableofcontents
}
\newpage

\subsubsection{1. Use the Spellman yeast cell cycle dataset
(spellman.txt).}\label{use-the-spellman-yeast-cell-cycle-dataset-spellman.txt.}

\begin{Shaded}
\begin{Highlighting}[]
\NormalTok{file <-}\StringTok{ "spellman.txt"}
\end{Highlighting}
\end{Shaded}

\subsubsection{\texorpdfstring{2. a) Read into R (Hint: using the
read.table() function with a ``header'' argument is one method to do
this).}{2. a) Read into R (Hint: using the read.table() function with a header argument is one method to do this).}}\label{a-read-into-r-hint-using-the-read.table-function-with-a-header-argument-is-one-method-to-do-this.}

\begin{Shaded}
\begin{Highlighting}[]
\NormalTok{spellman_data <-}\StringTok{ }\KeywordTok{read.table}\NormalTok{(file, }\DataTypeTok{header =} \NormalTok{T)}
\end{Highlighting}
\end{Shaded}

\subsubsection{2. b) Set the row names to the first column, then remove
this first
column.}\label{b-set-the-row-names-to-the-first-column-then-remove-this-first-column.}

\begin{Shaded}
\begin{Highlighting}[]
\KeywordTok{rownames}\NormalTok{(spellman_data) <-}\StringTok{ }\NormalTok{spellman_data$row.names}
\NormalTok{spellman_data$row.names <-}\StringTok{ }\OtherTok{NULL}
\end{Highlighting}
\end{Shaded}

\subsubsection{3. a) Look at the dimensions of the data frame and make
sure that there are 6,178 genes and 77
arrays/sample.}\label{a-look-at-the-dimensions-of-the-data-frame-and-make-sure-that-there-are-6178-genes-and-77-arrayssample.}

\begin{Shaded}
\begin{Highlighting}[]
\KeywordTok{dim}\NormalTok{(spellman_data)}
\end{Highlighting}
\end{Shaded}

\begin{verbatim}
## [1] 6178   77
\end{verbatim}

\newpage

\subsubsection{3. b) Isolate only the cdc15 experiment (samples 23-46),
pick a gene with some missing values (I use gene \#2/YAL002W in the
solutions), and impute with the row mean (save as
something).}\label{b-isolate-only-the-cdc15-experiment-samples-23-46-pick-a-gene-with-some-missing-values-i-use-gene-2yal002w-in-the-solutions-and-impute-with-the-row-mean-save-as-something.}

\begin{Shaded}
\begin{Highlighting}[]
\NormalTok{cdc15 <-}\StringTok{ }\KeywordTok{subset}\NormalTok{(spellman_data, }\DataTypeTok{select =} \KeywordTok{c}\NormalTok{(}\DecValTok{23}\NormalTok{:}\DecValTok{46}\NormalTok{))}
\NormalTok{cdc15[}\StringTok{"YAL004W"}\NormalTok{,]}
\end{Highlighting}
\end{Shaded}

\begin{verbatim}
##         cdc15_10 cdc15_30 cdc15_50 cdc15_70 cdc15_80 cdc15_90 cdc15_100
## YAL004W       NA       NA       NA     -1.5    -0.03     -1.2     -0.06
##         cdc15_110 cdc15_120 cdc15_130 cdc15_140 cdc15_150 cdc15_160
## YAL004W     -1.78      0.14     -1.13     -0.13     -1.27     -0.27
##         cdc15_170 cdc15_180 cdc15_190 cdc15_200 cdc15_210 cdc15_220
## YAL004W     -0.94      0.14        NA      1.04      0.48      1.94
##         cdc15_230 cdc15_240 cdc15_250 cdc15_270 cdc15_290
## YAL004W      1.62      1.73      1.22        NA        NA
\end{verbatim}

\begin{Shaded}
\begin{Highlighting}[]
\NormalTok{yal004w_mean <-}\StringTok{ }\KeywordTok{rowMeans}\NormalTok{(cdc15[}\StringTok{"YAL004W"}\NormalTok{,], }\DataTypeTok{na.rm =} \NormalTok{T)}
\NormalTok{yal004w_mean}
\end{Highlighting}
\end{Shaded}

\begin{verbatim}
##      YAL004W 
## 1.079383e-17
\end{verbatim}

\begin{Shaded}
\begin{Highlighting}[]
\NormalTok{for (x in }\DecValTok{1}\NormalTok{:}\KeywordTok{length}\NormalTok{(cdc15[}\StringTok{"YAL004W"}\NormalTok{,])) \{}
    \NormalTok{if (}\KeywordTok{is.na}\NormalTok{(cdc15[}\StringTok{"YAL004W"}\NormalTok{,x])) \{}
        \NormalTok{cdc15[}\StringTok{"YAL004W"}\NormalTok{,x] <-}\StringTok{ }\NormalTok{yal004w_mean}
    \NormalTok{\}}
\NormalTok{\}}
\NormalTok{yal <-}\StringTok{ }\NormalTok{cdc15[}\StringTok{"YAL004W"}\NormalTok{,]}
\NormalTok{yal}
\end{Highlighting}
\end{Shaded}

\begin{verbatim}
##             cdc15_10     cdc15_30     cdc15_50 cdc15_70 cdc15_80 cdc15_90
## YAL004W 1.079383e-17 1.079383e-17 1.079383e-17     -1.5    -0.03     -1.2
##         cdc15_100 cdc15_110 cdc15_120 cdc15_130 cdc15_140 cdc15_150
## YAL004W     -0.06     -1.78      0.14     -1.13     -0.13     -1.27
##         cdc15_160 cdc15_170 cdc15_180    cdc15_190 cdc15_200 cdc15_210
## YAL004W     -0.27     -0.94      0.14 1.079383e-17      1.04      0.48
##         cdc15_220 cdc15_230 cdc15_240 cdc15_250    cdc15_270    cdc15_290
## YAL004W      1.94      1.62      1.73      1.22 1.079383e-17 1.079383e-17
\end{verbatim}

\newpage

\subsubsection{4. Look up the functions for boxplot and hist and plot
the gene. Color the plots red and title them. Make sure that the vector
is numeric
(as.numeric).}\label{look-up-the-functions-for-boxplot-and-hist-and-plot-the-gene.-color-the-plots-red-and-title-them.-make-sure-that-the-vector-is-numeric-as.numeric.}

\begin{Shaded}
\begin{Highlighting}[]
\KeywordTok{boxplot}\NormalTok{(}\KeywordTok{as.numeric}\NormalTok{(yal), }\DataTypeTok{col =} \StringTok{"red"}\NormalTok{, }\DataTypeTok{main =} \StringTok{"Boxplot - Gene YAL004W"}\NormalTok{)}
\end{Highlighting}
\end{Shaded}

\includegraphics{lab2-7547611_files/figure-latex/unnamed-chunk-6-1.pdf}

\begin{Shaded}
\begin{Highlighting}[]
\KeywordTok{hist}\NormalTok{(}\KeywordTok{as.numeric}\NormalTok{(yal), }\DataTypeTok{freq =} \NormalTok{T, }\DataTypeTok{col =} \StringTok{"red"}\NormalTok{,}
    \DataTypeTok{main =} \StringTok{"Hist - Gene YAL004W"}\NormalTok{, }\DataTypeTok{xlab =} \StringTok{""}\NormalTok{)}
\end{Highlighting}
\end{Shaded}

\includegraphics{lab2-7547611_files/figure-latex/unnamed-chunk-6-2.pdf}
\newpage

\subsubsection{5. Generate a profile plot of the same gene. Title the
plot. Use lwd in the plot command (lwd=line
width).}\label{generate-a-profile-plot-of-the-same-gene.-title-the-plot.-use-lwd-in-the-plot-command-lwdline-width.}

\begin{Shaded}
\begin{Highlighting}[]
\KeywordTok{plot}\NormalTok{(}
    \DataTypeTok{x =} \DecValTok{1}\NormalTok{:}\KeywordTok{length}\NormalTok{(yal),}
    \DataTypeTok{y =} \KeywordTok{as.numeric}\NormalTok{(yal),}
    \DataTypeTok{type =} \StringTok{"o"}\NormalTok{,}
    \DataTypeTok{lwd =} \DecValTok{1}\NormalTok{,}
    \DataTypeTok{main =} \StringTok{"Profile of YAL004W"}\NormalTok{,}
    \DataTypeTok{col =} \StringTok{"red"}\NormalTok{,}
    \DataTypeTok{xlab =} \StringTok{""}\NormalTok{,}
    \DataTypeTok{ylab =} \StringTok{""}
\NormalTok{)}
\end{Highlighting}
\end{Shaded}

\includegraphics{lab2-7547611_files/figure-latex/unnamed-chunk-7-1.pdf}

\end{document}
