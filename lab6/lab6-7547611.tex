\documentclass[]{article}
\usepackage{lmodern}
\usepackage{amssymb,amsmath}
\usepackage{ifxetex,ifluatex}
\usepackage{fixltx2e} % provides \textsubscript
\ifnum 0\ifxetex 1\fi\ifluatex 1\fi=0 % if pdftex
  \usepackage[T1]{fontenc}
  \usepackage[utf8]{inputenc}
\else % if luatex or xelatex
  \ifxetex
    \usepackage{mathspec}
  \else
    \usepackage{fontspec}
  \fi
  \defaultfontfeatures{Ligatures=TeX,Scale=MatchLowercase}
  \newcommand{\euro}{€}
\fi
% use upquote if available, for straight quotes in verbatim environments
\IfFileExists{upquote.sty}{\usepackage{upquote}}{}
% use microtype if available
\IfFileExists{microtype.sty}{%
\usepackage{microtype}
\UseMicrotypeSet[protrusion]{basicmath} % disable protrusion for tt fonts
}{}
\usepackage[margin=1in]{geometry}
\usepackage{hyperref}
\PassOptionsToPackage{usenames,dvipsnames}{color} % color is loaded by hyperref
\hypersetup{unicode=true,
            pdftitle={Lab6},
            pdfauthor={Jessica Temporal 7547611},
            pdfborder={0 0 0},
            breaklinks=true}
\urlstyle{same}  % don't use monospace font for urls
\usepackage{color}
\usepackage{fancyvrb}
\newcommand{\VerbBar}{|}
\newcommand{\VERB}{\Verb[commandchars=\\\{\}]}
\DefineVerbatimEnvironment{Highlighting}{Verbatim}{commandchars=\\\{\}}
% Add ',fontsize=\small' for more characters per line
\usepackage{framed}
\definecolor{shadecolor}{RGB}{248,248,248}
\newenvironment{Shaded}{\begin{snugshade}}{\end{snugshade}}
\newcommand{\KeywordTok}[1]{\textcolor[rgb]{0.13,0.29,0.53}{\textbf{{#1}}}}
\newcommand{\DataTypeTok}[1]{\textcolor[rgb]{0.13,0.29,0.53}{{#1}}}
\newcommand{\DecValTok}[1]{\textcolor[rgb]{0.00,0.00,0.81}{{#1}}}
\newcommand{\BaseNTok}[1]{\textcolor[rgb]{0.00,0.00,0.81}{{#1}}}
\newcommand{\FloatTok}[1]{\textcolor[rgb]{0.00,0.00,0.81}{{#1}}}
\newcommand{\ConstantTok}[1]{\textcolor[rgb]{0.00,0.00,0.00}{{#1}}}
\newcommand{\CharTok}[1]{\textcolor[rgb]{0.31,0.60,0.02}{{#1}}}
\newcommand{\SpecialCharTok}[1]{\textcolor[rgb]{0.00,0.00,0.00}{{#1}}}
\newcommand{\StringTok}[1]{\textcolor[rgb]{0.31,0.60,0.02}{{#1}}}
\newcommand{\VerbatimStringTok}[1]{\textcolor[rgb]{0.31,0.60,0.02}{{#1}}}
\newcommand{\SpecialStringTok}[1]{\textcolor[rgb]{0.31,0.60,0.02}{{#1}}}
\newcommand{\ImportTok}[1]{{#1}}
\newcommand{\CommentTok}[1]{\textcolor[rgb]{0.56,0.35,0.01}{\textit{{#1}}}}
\newcommand{\DocumentationTok}[1]{\textcolor[rgb]{0.56,0.35,0.01}{\textbf{\textit{{#1}}}}}
\newcommand{\AnnotationTok}[1]{\textcolor[rgb]{0.56,0.35,0.01}{\textbf{\textit{{#1}}}}}
\newcommand{\CommentVarTok}[1]{\textcolor[rgb]{0.56,0.35,0.01}{\textbf{\textit{{#1}}}}}
\newcommand{\OtherTok}[1]{\textcolor[rgb]{0.56,0.35,0.01}{{#1}}}
\newcommand{\FunctionTok}[1]{\textcolor[rgb]{0.00,0.00,0.00}{{#1}}}
\newcommand{\VariableTok}[1]{\textcolor[rgb]{0.00,0.00,0.00}{{#1}}}
\newcommand{\ControlFlowTok}[1]{\textcolor[rgb]{0.13,0.29,0.53}{\textbf{{#1}}}}
\newcommand{\OperatorTok}[1]{\textcolor[rgb]{0.81,0.36,0.00}{\textbf{{#1}}}}
\newcommand{\BuiltInTok}[1]{{#1}}
\newcommand{\ExtensionTok}[1]{{#1}}
\newcommand{\PreprocessorTok}[1]{\textcolor[rgb]{0.56,0.35,0.01}{\textit{{#1}}}}
\newcommand{\AttributeTok}[1]{\textcolor[rgb]{0.77,0.63,0.00}{{#1}}}
\newcommand{\RegionMarkerTok}[1]{{#1}}
\newcommand{\InformationTok}[1]{\textcolor[rgb]{0.56,0.35,0.01}{\textbf{\textit{{#1}}}}}
\newcommand{\WarningTok}[1]{\textcolor[rgb]{0.56,0.35,0.01}{\textbf{\textit{{#1}}}}}
\newcommand{\AlertTok}[1]{\textcolor[rgb]{0.94,0.16,0.16}{{#1}}}
\newcommand{\ErrorTok}[1]{\textcolor[rgb]{0.64,0.00,0.00}{\textbf{{#1}}}}
\newcommand{\NormalTok}[1]{{#1}}
\usepackage{graphicx,grffile}
\makeatletter
\def\maxwidth{\ifdim\Gin@nat@width>\linewidth\linewidth\else\Gin@nat@width\fi}
\def\maxheight{\ifdim\Gin@nat@height>\textheight\textheight\else\Gin@nat@height\fi}
\makeatother
% Scale images if necessary, so that they will not overflow the page
% margins by default, and it is still possible to overwrite the defaults
% using explicit options in \includegraphics[width, height, ...]{}
\setkeys{Gin}{width=\maxwidth,height=\maxheight,keepaspectratio}
\setlength{\parindent}{0pt}
\setlength{\parskip}{6pt plus 2pt minus 1pt}
\setlength{\emergencystretch}{3em}  % prevent overfull lines
\providecommand{\tightlist}{%
  \setlength{\itemsep}{0pt}\setlength{\parskip}{0pt}}
\setcounter{secnumdepth}{0}

%%% Use protect on footnotes to avoid problems with footnotes in titles
\let\rmarkdownfootnote\footnote%
\def\footnote{\protect\rmarkdownfootnote}

%%% Change title format to be more compact
\usepackage{titling}

% Create subtitle command for use in maketitle
\newcommand{\subtitle}[1]{
  \posttitle{
    \begin{center}\large#1\end{center}
    }
}

\setlength{\droptitle}{-2em}
  \title{Lab6}
  \pretitle{\vspace{\droptitle}\centering\huge}
  \posttitle{\par}
  \author{Jessica Temporal 7547611}
  \preauthor{\centering\large\emph}
  \postauthor{\par}
  \predate{\centering\large\emph}
  \postdate{\par}
  \date{September 29, 2016}



% Redefines (sub)paragraphs to behave more like sections
\ifx\paragraph\undefined\else
\let\oldparagraph\paragraph
\renewcommand{\paragraph}[1]{\oldparagraph{#1}\mbox{}}
\fi
\ifx\subparagraph\undefined\else
\let\oldsubparagraph\subparagraph
\renewcommand{\subparagraph}[1]{\oldsubparagraph{#1}\mbox{}}
\fi

\begin{document}
\maketitle

{
\setcounter{tocdepth}{4}
\tableofcontents
}
\newpage

\subsubsection{1. Get the GEO Brain Aging study
(agingStudy11FCortexAffy.txt, agingStudy11FCortexAffyAnn.txt). Also
obtain the annotation file for this data
frame.}\label{get-the-geo-brain-aging-study-agingstudy11fcortexaffy.txt-agingstudy11fcortexaffyann.txt.-also-obtain-the-annotation-file-for-this-data-frame.}

\begin{Shaded}
\begin{Highlighting}[]
\NormalTok{study_file <-}\StringTok{ "agingStudy11FCortexAffy.txt"}
\NormalTok{anno_file <-}\StringTok{ "agingStudy1FCortexAffyAnn.txt"}
\end{Highlighting}
\end{Shaded}

\subsubsection{\texorpdfstring{2. Load into R, using
\texttt{read.table()} function and the header=T/row.names=1 arguments
for each data
file.}{2. Load into R, using read.table() function and the header=T/row.names=1 arguments for each data file.}}\label{load-into-r-using-read.table-function-and-the-headertrow.names1-arguments-for-each-data-file.}

\begin{Shaded}
\begin{Highlighting}[]
\NormalTok{study <-}\StringTok{ }\KeywordTok{read.table}\NormalTok{(study_file, }\DataTypeTok{header =} \NormalTok{T, }\DataTypeTok{na.strings =} \StringTok{"NA"}\NormalTok{, }\DataTypeTok{blank.lines.skip =} \NormalTok{F, }\DataTypeTok{row.names =} \DecValTok{1}\NormalTok{)}
\NormalTok{study_ann <-}\StringTok{ }\KeywordTok{read.table}\NormalTok{(anno_file, }\DataTypeTok{header =} \NormalTok{T, }\DataTypeTok{na.strings =} \StringTok{"NA"}\NormalTok{, }\DataTypeTok{blank.lines.skip =} \NormalTok{F, }\DataTypeTok{row.names =} \DecValTok{1}\NormalTok{)}
\end{Highlighting}
\end{Shaded}

\subsubsection{3. Prepare 2 separate vectors for comparison. The first
is a comparison between male and female patients. The current data frame
can be left alone for this, since the males and females are all grouped
together. The second vector is comparison between patients
\textgreater{}= 50 years of age and those \textless{} 50 years of age.
To do this, you must use the annotation file and logical operators to
isolate the correct
arrays/samples.}\label{prepare-2-separate-vectors-for-comparison.-the-first-is-a-comparison-between-male-and-female-patients.-the-current-data-frame-can-be-left-alone-for-this-since-the-males-and-females-are-all-grouped-together.-the-second-vector-is-comparison-between-patients-50-years-of-age-and-those-50-years-of-age.-to-do-this-you-must-use-the-annotation-file-and-logical-operators-to-isolate-the-correct-arrayssamples.}

\begin{Shaded}
\begin{Highlighting}[]
\NormalTok{study_names <-}\StringTok{ }\KeywordTok{data.frame}\NormalTok{(}\DataTypeTok{names =} \KeywordTok{dimnames}\NormalTok{(study)[[}\DecValTok{2}\NormalTok{]])}
\NormalTok{study_names$sample <-}\StringTok{ }\OtherTok{NA}
\NormalTok{study_names$sex <-}\StringTok{ }\OtherTok{NA}

\NormalTok{for(i in }\DecValTok{1}\NormalTok{:}\DecValTok{30}\NormalTok{) \{}
    \NormalTok{study_names$sample[i] <-}\StringTok{ }\KeywordTok{strsplit}\NormalTok{(}\KeywordTok{as.character}\NormalTok{(study_names$names), }\StringTok{'[.]'}\NormalTok{)[[i]][}\DecValTok{1}\NormalTok{]}
    \NormalTok{study_names$sex[i] <-}\StringTok{ }\KeywordTok{strsplit}\NormalTok{(}\KeywordTok{as.character}\NormalTok{(study_names$names), }\StringTok{'[.]'}\NormalTok{)[[i]][}\DecValTok{3}\NormalTok{]}
\NormalTok{\}}
\KeywordTok{colnames}\NormalTok{(study) <-}\StringTok{ }\NormalTok{study_names$sample}

\NormalTok{study_m <-}\StringTok{ }\NormalTok{study[,}\KeywordTok{subset}\NormalTok{(study_names, }\DataTypeTok{subset =} \NormalTok{sex==}\StringTok{'M'}\NormalTok{, }\DataTypeTok{select =} \StringTok{'sample'}\NormalTok{)$sample]}
\NormalTok{study_f <-}\StringTok{ }\NormalTok{study[,}\KeywordTok{subset}\NormalTok{(study_names, }\DataTypeTok{subset =} \NormalTok{sex==}\StringTok{'F'}\NormalTok{, }\DataTypeTok{select =} \StringTok{'sample'}\NormalTok{)$sample]}

\NormalTok{study_lower <-}\StringTok{ }\NormalTok{study[,}\KeywordTok{rownames}\NormalTok{(}\KeywordTok{subset}\NormalTok{(study_ann, }\DataTypeTok{subset =} \NormalTok{Age <}\StringTok{ }\DecValTok{50}\NormalTok{))]}
\NormalTok{study_higher <-}\StringTok{ }\NormalTok{study[,}\KeywordTok{rownames}\NormalTok{(}\KeywordTok{subset}\NormalTok{(study_ann, }\DataTypeTok{subset =} \NormalTok{Age >=}\StringTok{ }\DecValTok{50}\NormalTok{))]}
\end{Highlighting}
\end{Shaded}

\subsubsection{4. Run the t.test function from the notes using the first
gene vector below for the gender comparison. Then use the second gene
vector below for the age comparison. Using these p-values, use either
p.adjust in the base library or mt.rawp2adjp in the multitest library to
adjust the values for multiple corrections with the Holm's
method.}\label{run-the-t.test-function-from-the-notes-using-the-first-gene-vector-below-for-the-gender-comparison.-then-use-the-second-gene-vector-below-for-the-age-comparison.-using-these-p-values-use-either-p.adjust-in-the-base-library-or-mt.rawp2adjp-in-the-multitest-library-to-adjust-the-values-for-multiple-corrections-with-the-holms-method.}

\begin{Shaded}
\begin{Highlighting}[]
\CommentTok{# gender comparison gene vector}
\NormalTok{g.g <-}\StringTok{ }\KeywordTok{c}\NormalTok{(}\DecValTok{1394}\NormalTok{,  }\DecValTok{1474}\NormalTok{,  }\DecValTok{1917}\NormalTok{,  }\DecValTok{2099}\NormalTok{,  }\DecValTok{2367}\NormalTok{,  }\DecValTok{2428}\NormalTok{, }\DecValTok{2625}\NormalTok{,  }\DecValTok{3168}\NormalTok{,  }\DecValTok{3181}\NormalTok{,  }\DecValTok{3641}\NormalTok{,  }\DecValTok{3832}\NormalTok{,  }\DecValTok{4526}\NormalTok{,}
\DecValTok{4731}\NormalTok{,  }\DecValTok{4863}\NormalTok{,  }\DecValTok{6062}\NormalTok{,  }\DecValTok{6356}\NormalTok{,  }\DecValTok{6684}\NormalTok{,  }\DecValTok{6787}\NormalTok{,  }\DecValTok{6900}\NormalTok{,  }\DecValTok{7223}\NormalTok{,  }\DecValTok{7244}\NormalTok{,  }\DecValTok{7299}\NormalTok{,  }\DecValTok{8086}\NormalTok{,  }\DecValTok{8652}\NormalTok{,}
\DecValTok{8959}\NormalTok{,  }\DecValTok{9073}\NormalTok{,  }\DecValTok{9145}\NormalTok{,  }\DecValTok{9389}\NormalTok{, }\DecValTok{10219}\NormalTok{, }\DecValTok{11238}\NormalTok{, }\DecValTok{11669}\NormalTok{, }\DecValTok{11674}\NormalTok{, }\DecValTok{11793}\NormalTok{)}

\CommentTok{# age comparison gene vector}
\NormalTok{g.a <-}\StringTok{ }\KeywordTok{c}\NormalTok{(}\DecValTok{25}\NormalTok{, }\DecValTok{302}\NormalTok{,  }\DecValTok{1847}\NormalTok{,  }\DecValTok{2324}\NormalTok{,  }\DecValTok{246}\NormalTok{,  }\DecValTok{2757}\NormalTok{, }\DecValTok{3222}\NormalTok{, }\DecValTok{3675}\NormalTok{,  }\DecValTok{4429}\NormalTok{,  }\DecValTok{4430}\NormalTok{,  }\DecValTok{4912}\NormalTok{,  }\DecValTok{5640}\NormalTok{, }\DecValTok{5835}\NormalTok{, }\DecValTok{5856}\NormalTok{,  }\DecValTok{6803}\NormalTok{,  }\DecValTok{7229}\NormalTok{,  }\DecValTok{7833}\NormalTok{,  }\DecValTok{8133}\NormalTok{, }\DecValTok{8579}\NormalTok{,  }\DecValTok{8822}\NormalTok{,  }\DecValTok{8994}\NormalTok{, }\DecValTok{10101}\NormalTok{, }\DecValTok{11433}\NormalTok{, }\DecValTok{12039}\NormalTok{, }\DecValTok{12353}\NormalTok{,}
\DecValTok{12404}\NormalTok{, }\DecValTok{12442}\NormalTok{, }\DecValTok{67}\NormalTok{, }\DecValTok{88}\NormalTok{, }\DecValTok{100}\NormalTok{)}

\CommentTok{# t.test function from the notes}
\NormalTok{t.test.all.genes <-}\StringTok{ }\NormalTok{function(x,s1,s2) \{}
    \NormalTok{x1 <-}\StringTok{ }\NormalTok{x[s1]}
    \NormalTok{x2 <-}\StringTok{ }\NormalTok{x[s2]}
    \NormalTok{x1 <-}\StringTok{ }\KeywordTok{as.numeric}\NormalTok{(x1)}
    \NormalTok{x2 <-}\StringTok{ }\KeywordTok{as.numeric}\NormalTok{(x2)}
    \NormalTok{t.out <-}\StringTok{ }\KeywordTok{t.test}\NormalTok{(x1,x2, }\DataTypeTok{alternative=}\StringTok{"two.sided"}\NormalTok{,}\DataTypeTok{var.equal=}\NormalTok{T)}
    \NormalTok{out <-}\StringTok{ }\KeywordTok{as.numeric}\NormalTok{(t.out$p.value)}
    \KeywordTok{return}\NormalTok{(out)}
\NormalTok{\}}

\NormalTok{gender_test <-}\StringTok{ }\KeywordTok{apply}\NormalTok{(study[g.g,], }\DecValTok{1}\NormalTok{, t.test.all.genes, }\DataTypeTok{s1 =} \KeywordTok{as.matrix}\NormalTok{(study_m), }\DataTypeTok{s2 =} \KeywordTok{as.matrix}\NormalTok{(study_f))}
\NormalTok{age_test <-}\StringTok{ }\KeywordTok{apply}\NormalTok{(study[g.a,], }\DecValTok{1}\NormalTok{, t.test.all.genes, }\DataTypeTok{s1 =} \KeywordTok{as.matrix}\NormalTok{(study_higher), }\DataTypeTok{s2 =} \KeywordTok{as.matrix}\NormalTok{(study_lower))}

\NormalTok{gender_test_adjusted <-}\StringTok{ }\KeywordTok{p.adjust}\NormalTok{(gender_test, }\DataTypeTok{method =} \StringTok{"holm"}\NormalTok{)}
\NormalTok{age_test_adjusted <-}\StringTok{ }\KeywordTok{p.adjust}\NormalTok{(age_test, }\DataTypeTok{method =} \StringTok{"holm"}\NormalTok{)}
\end{Highlighting}
\end{Shaded}

\subsubsection{\texorpdfstring{5. Sort the adjusted p-values and
non-adjusted p-values and plot them vs.~the x-axis of numbers (e.g.
\texttt{1-length(p.adj)}) for each comparison data set. Make sure that
the two lines are different colors. Hint: use \texttt{sort()} to sort
the
values.}{5. Sort the adjusted p-values and non-adjusted p-values and plot them vs.~the x-axis of numbers (e.g. 1-length(p.adj)) for each comparison data set. Make sure that the two lines are different colors. Hint: use sort() to sort the values.}}\label{sort-the-adjusted-p-values-and-non-adjusted-p-values-and-plot-them-vs.the-x-axis-of-numbers-e.g.-1-lengthp.adj-for-each-comparison-data-set.-make-sure-that-the-two-lines-are-different-colors.-hint-use-sort-to-sort-the-values.}

\begin{Shaded}
\begin{Highlighting}[]
\KeywordTok{plot}\NormalTok{(}\KeywordTok{sort}\NormalTok{(gender_test), }\DataTypeTok{col =} \StringTok{"red"}\NormalTok{, }\DataTypeTok{type =} \StringTok{"l"}\NormalTok{, }
     \DataTypeTok{main =} \StringTok{"Non-Adjusted Gender vs Adjusted Gender}\CharTok{\textbackslash{}n}\StringTok{holm"}\NormalTok{, }\DataTypeTok{ylim =} \KeywordTok{c}\NormalTok{(}\DecValTok{0}\NormalTok{,}\FloatTok{1.1}\NormalTok{))}
\KeywordTok{points}\NormalTok{(}\KeywordTok{sort}\NormalTok{(gender_test_adjusted), }\DataTypeTok{col =} \StringTok{"blue"}\NormalTok{, }\DataTypeTok{type =} \StringTok{"l"}\NormalTok{)}
\end{Highlighting}
\end{Shaded}

\includegraphics{lab6-7547611_files/figure-latex/unnamed-chunk-5-1.pdf}
\newpage

\begin{Shaded}
\begin{Highlighting}[]
\KeywordTok{plot}\NormalTok{(}\KeywordTok{sort}\NormalTok{(age_test), }\DataTypeTok{col =} \StringTok{"red"}\NormalTok{, }\DataTypeTok{type =} \StringTok{"l"}\NormalTok{, }
     \DataTypeTok{main =} \StringTok{"Non-Adjusted Age vs Adjusted Age}\CharTok{\textbackslash{}n}\StringTok{holm"}\NormalTok{, }\DataTypeTok{ylim =} \KeywordTok{c}\NormalTok{(}\DecValTok{0}\NormalTok{,}\FloatTok{1.1}\NormalTok{))}
\KeywordTok{points}\NormalTok{(}\KeywordTok{sort}\NormalTok{(age_test_adjusted), }\DataTypeTok{col =} \StringTok{"blue"}\NormalTok{, }\DataTypeTok{type =} \StringTok{"l"}\NormalTok{)}
\end{Highlighting}
\end{Shaded}

\includegraphics{lab6-7547611_files/figure-latex/unnamed-chunk-6-1.pdf}
\newpage

\subsubsection{6. Repeat \#4 and \#5 with the Bonferroni
method.}\label{repeat-4-and-5-with-the-bonferroni-method.}

\begin{Shaded}
\begin{Highlighting}[]
\NormalTok{bon_gender_test_adjusted <-}\StringTok{ }\KeywordTok{p.adjust}\NormalTok{(gender_test, }\DataTypeTok{method =} \StringTok{"bonferroni"}\NormalTok{)}
\NormalTok{bon_age_test_adjusted <-}\StringTok{ }\KeywordTok{p.adjust}\NormalTok{(age_test, }\DataTypeTok{method =} \StringTok{"bonferroni"}\NormalTok{)}

\KeywordTok{plot}\NormalTok{(}\KeywordTok{sort}\NormalTok{(gender_test), }\DataTypeTok{col =} \StringTok{"red"}\NormalTok{, }\DataTypeTok{type =} \StringTok{"l"}\NormalTok{, }
     \DataTypeTok{main =} \StringTok{"Non-Adjusted Gender vs Adjusted Gender}\CharTok{\textbackslash{}n}\StringTok{Bonferroni"}\NormalTok{, }\DataTypeTok{ylim =} \KeywordTok{c}\NormalTok{(}\DecValTok{0}\NormalTok{,}\FloatTok{1.1}\NormalTok{))}
\KeywordTok{points}\NormalTok{(}\KeywordTok{sort}\NormalTok{(bon_gender_test_adjusted), }\DataTypeTok{col =} \StringTok{"blue"}\NormalTok{, }\DataTypeTok{type =} \StringTok{"l"}\NormalTok{)}
\end{Highlighting}
\end{Shaded}

\includegraphics{lab6-7547611_files/figure-latex/unnamed-chunk-7-1.pdf}
\newpage

\begin{Shaded}
\begin{Highlighting}[]
\KeywordTok{plot}\NormalTok{(}\KeywordTok{sort}\NormalTok{(age_test), }\DataTypeTok{col =} \StringTok{"red"}\NormalTok{, }\DataTypeTok{type =} \StringTok{"l"}\NormalTok{,}
     \DataTypeTok{main =} \StringTok{"Non-Adjusted Age vs Adjusted Age}\CharTok{\textbackslash{}n}\StringTok{Bonferroni"}\NormalTok{, }\DataTypeTok{ylim =} \KeywordTok{c}\NormalTok{(}\DecValTok{0}\NormalTok{,}\FloatTok{1.1}\NormalTok{))}
\KeywordTok{points}\NormalTok{(}\KeywordTok{sort}\NormalTok{(bon_age_test_adjusted), }\DataTypeTok{col =} \StringTok{"blue"}\NormalTok{, }\DataTypeTok{type =} \StringTok{"l"}\NormalTok{)}
\end{Highlighting}
\end{Shaded}

\includegraphics{lab6-7547611_files/figure-latex/unnamed-chunk-8-1.pdf}

\end{document}
