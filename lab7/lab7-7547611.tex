\documentclass[]{article}
\usepackage{lmodern}
\usepackage{amssymb,amsmath}
\usepackage{ifxetex,ifluatex}
\usepackage{fixltx2e} % provides \textsubscript
\ifnum 0\ifxetex 1\fi\ifluatex 1\fi=0 % if pdftex
  \usepackage[T1]{fontenc}
  \usepackage[utf8]{inputenc}
\else % if luatex or xelatex
  \ifxetex
    \usepackage{mathspec}
  \else
    \usepackage{fontspec}
  \fi
  \defaultfontfeatures{Ligatures=TeX,Scale=MatchLowercase}
  \newcommand{\euro}{€}
\fi
% use upquote if available, for straight quotes in verbatim environments
\IfFileExists{upquote.sty}{\usepackage{upquote}}{}
% use microtype if available
\IfFileExists{microtype.sty}{%
\usepackage{microtype}
\UseMicrotypeSet[protrusion]{basicmath} % disable protrusion for tt fonts
}{}
\usepackage[margin=1in]{geometry}
\usepackage{hyperref}
\PassOptionsToPackage{usenames,dvipsnames}{color} % color is loaded by hyperref
\hypersetup{unicode=true,
            pdftitle={Lab7},
            pdfauthor={Jessica Temporal 7547611},
            pdfsubject={Cluster Analysis},
            pdfborder={0 0 0},
            breaklinks=true}
\urlstyle{same}  % don't use monospace font for urls
\usepackage{color}
\usepackage{fancyvrb}
\newcommand{\VerbBar}{|}
\newcommand{\VERB}{\Verb[commandchars=\\\{\}]}
\DefineVerbatimEnvironment{Highlighting}{Verbatim}{commandchars=\\\{\}}
% Add ',fontsize=\small' for more characters per line
\usepackage{framed}
\definecolor{shadecolor}{RGB}{248,248,248}
\newenvironment{Shaded}{\begin{snugshade}}{\end{snugshade}}
\newcommand{\KeywordTok}[1]{\textcolor[rgb]{0.13,0.29,0.53}{\textbf{{#1}}}}
\newcommand{\DataTypeTok}[1]{\textcolor[rgb]{0.13,0.29,0.53}{{#1}}}
\newcommand{\DecValTok}[1]{\textcolor[rgb]{0.00,0.00,0.81}{{#1}}}
\newcommand{\BaseNTok}[1]{\textcolor[rgb]{0.00,0.00,0.81}{{#1}}}
\newcommand{\FloatTok}[1]{\textcolor[rgb]{0.00,0.00,0.81}{{#1}}}
\newcommand{\ConstantTok}[1]{\textcolor[rgb]{0.00,0.00,0.00}{{#1}}}
\newcommand{\CharTok}[1]{\textcolor[rgb]{0.31,0.60,0.02}{{#1}}}
\newcommand{\SpecialCharTok}[1]{\textcolor[rgb]{0.00,0.00,0.00}{{#1}}}
\newcommand{\StringTok}[1]{\textcolor[rgb]{0.31,0.60,0.02}{{#1}}}
\newcommand{\VerbatimStringTok}[1]{\textcolor[rgb]{0.31,0.60,0.02}{{#1}}}
\newcommand{\SpecialStringTok}[1]{\textcolor[rgb]{0.31,0.60,0.02}{{#1}}}
\newcommand{\ImportTok}[1]{{#1}}
\newcommand{\CommentTok}[1]{\textcolor[rgb]{0.56,0.35,0.01}{\textit{{#1}}}}
\newcommand{\DocumentationTok}[1]{\textcolor[rgb]{0.56,0.35,0.01}{\textbf{\textit{{#1}}}}}
\newcommand{\AnnotationTok}[1]{\textcolor[rgb]{0.56,0.35,0.01}{\textbf{\textit{{#1}}}}}
\newcommand{\CommentVarTok}[1]{\textcolor[rgb]{0.56,0.35,0.01}{\textbf{\textit{{#1}}}}}
\newcommand{\OtherTok}[1]{\textcolor[rgb]{0.56,0.35,0.01}{{#1}}}
\newcommand{\FunctionTok}[1]{\textcolor[rgb]{0.00,0.00,0.00}{{#1}}}
\newcommand{\VariableTok}[1]{\textcolor[rgb]{0.00,0.00,0.00}{{#1}}}
\newcommand{\ControlFlowTok}[1]{\textcolor[rgb]{0.13,0.29,0.53}{\textbf{{#1}}}}
\newcommand{\OperatorTok}[1]{\textcolor[rgb]{0.81,0.36,0.00}{\textbf{{#1}}}}
\newcommand{\BuiltInTok}[1]{{#1}}
\newcommand{\ExtensionTok}[1]{{#1}}
\newcommand{\PreprocessorTok}[1]{\textcolor[rgb]{0.56,0.35,0.01}{\textit{{#1}}}}
\newcommand{\AttributeTok}[1]{\textcolor[rgb]{0.77,0.63,0.00}{{#1}}}
\newcommand{\RegionMarkerTok}[1]{{#1}}
\newcommand{\InformationTok}[1]{\textcolor[rgb]{0.56,0.35,0.01}{\textbf{\textit{{#1}}}}}
\newcommand{\WarningTok}[1]{\textcolor[rgb]{0.56,0.35,0.01}{\textbf{\textit{{#1}}}}}
\newcommand{\AlertTok}[1]{\textcolor[rgb]{0.94,0.16,0.16}{{#1}}}
\newcommand{\ErrorTok}[1]{\textcolor[rgb]{0.64,0.00,0.00}{\textbf{{#1}}}}
\newcommand{\NormalTok}[1]{{#1}}
\usepackage{graphicx,grffile}
\makeatletter
\def\maxwidth{\ifdim\Gin@nat@width>\linewidth\linewidth\else\Gin@nat@width\fi}
\def\maxheight{\ifdim\Gin@nat@height>\textheight\textheight\else\Gin@nat@height\fi}
\makeatother
% Scale images if necessary, so that they will not overflow the page
% margins by default, and it is still possible to overwrite the defaults
% using explicit options in \includegraphics[width, height, ...]{}
\setkeys{Gin}{width=\maxwidth,height=\maxheight,keepaspectratio}
\setlength{\parindent}{0pt}
\setlength{\parskip}{6pt plus 2pt minus 1pt}
\setlength{\emergencystretch}{3em}  % prevent overfull lines
\providecommand{\tightlist}{%
  \setlength{\itemsep}{0pt}\setlength{\parskip}{0pt}}
\setcounter{secnumdepth}{0}

%%% Use protect on footnotes to avoid problems with footnotes in titles
\let\rmarkdownfootnote\footnote%
\def\footnote{\protect\rmarkdownfootnote}

%%% Change title format to be more compact
\usepackage{titling}

% Create subtitle command for use in maketitle
\newcommand{\subtitle}[1]{
  \posttitle{
    \begin{center}\large#1\end{center}
    }
}

\setlength{\droptitle}{-2em}
  \title{Lab7}
  \pretitle{\vspace{\droptitle}\centering\huge}
  \posttitle{\par}
\subtitle{Cluster Analysis}
  \author{Jessica Temporal 7547611}
  \preauthor{\centering\large\emph}
  \postauthor{\par}
  \predate{\centering\large\emph}
  \postdate{\par}
  \date{October 6, 2016}



% Redefines (sub)paragraphs to behave more like sections
\ifx\paragraph\undefined\else
\let\oldparagraph\paragraph
\renewcommand{\paragraph}[1]{\oldparagraph{#1}\mbox{}}
\fi
\ifx\subparagraph\undefined\else
\let\oldsubparagraph\subparagraph
\renewcommand{\subparagraph}[1]{\oldsubparagraph{#1}\mbox{}}
\fi

\begin{document}
\maketitle

{
\setcounter{tocdepth}{4}
\tableofcontents
}
\newpage

\subsubsection{\texorpdfstring{1. Load the fibroEset library and data
set (\texttt{library(fibroEset)}). Obtain the classifications for the
samples.}{1. Load the fibroEset library and data set (library(fibroEset)). Obtain the classifications for the samples.}}\label{load-the-fibroeset-library-and-data-set-libraryfibroeset.-obtain-the-classifications-for-the-samples.}

\begin{Shaded}
\begin{Highlighting}[]
\KeywordTok{source}\NormalTok{(}\StringTok{"http://bioconductor.org/biocLite.R"}\NormalTok{)}
\end{Highlighting}
\end{Shaded}

\begin{verbatim}
## Bioconductor version 3.2 (BiocInstaller 1.20.3), ?biocLite for help
\end{verbatim}

\begin{verbatim}
## A new version of Bioconductor is available after installing the most
##   recent version of R; see http://bioconductor.org/install
\end{verbatim}

\begin{Shaded}
\begin{Highlighting}[]
\KeywordTok{biocLite}\NormalTok{(}\StringTok{"fibroEset"}\NormalTok{)}
\end{Highlighting}
\end{Shaded}

\begin{verbatim}
## BioC_mirror: https://bioconductor.org
\end{verbatim}

\begin{verbatim}
## Using Bioconductor 3.2 (BiocInstaller 1.20.3), R 3.2.2 (2015-08-14).
\end{verbatim}

\begin{verbatim}
## Installing package(s) 'fibroEset'
\end{verbatim}

\begin{verbatim}
## 
## The downloaded source packages are in
##  '/tmp/RtmpXvMORv/downloaded_packages'
\end{verbatim}

\begin{verbatim}
## Old packages: 'boot', 'class', 'cluster', 'codetools', 'foreign',
##   'lattice', 'MASS', 'Matrix', 'mgcv', 'nlme', 'nnet', 'spatial',
##   'survival'
\end{verbatim}

\begin{Shaded}
\begin{Highlighting}[]
\KeywordTok{library}\NormalTok{(fibroEset)}
\end{Highlighting}
\end{Shaded}

\begin{verbatim}
## Loading required package: Biobase
\end{verbatim}

\begin{verbatim}
## Loading required package: BiocGenerics
\end{verbatim}

\begin{verbatim}
## Loading required package: parallel
\end{verbatim}

\begin{verbatim}
## 
## Attaching package: 'BiocGenerics'
\end{verbatim}

\begin{verbatim}
## The following objects are masked from 'package:parallel':
## 
##     clusterApply, clusterApplyLB, clusterCall, clusterEvalQ,
##     clusterExport, clusterMap, parApply, parCapply, parLapply,
##     parLapplyLB, parRapply, parSapply, parSapplyLB
\end{verbatim}

\begin{verbatim}
## The following objects are masked from 'package:stats':
## 
##     IQR, mad, xtabs
\end{verbatim}

\begin{verbatim}
## The following objects are masked from 'package:base':
## 
##     anyDuplicated, append, as.data.frame, as.vector, cbind,
##     colnames, do.call, duplicated, eval, evalq, Filter, Find, get,
##     grep, grepl, intersect, is.unsorted, lapply, lengths, Map,
##     mapply, match, mget, order, paste, pmax, pmax.int, pmin,
##     pmin.int, Position, rank, rbind, Reduce, rownames, sapply,
##     setdiff, sort, table, tapply, union, unique, unlist, unsplit
\end{verbatim}

\begin{verbatim}
## Welcome to Bioconductor
## 
##     Vignettes contain introductory material; view with
##     'browseVignettes()'. To cite Bioconductor, see
##     'citation("Biobase")', and for packages 'citation("pkgname")'.
\end{verbatim}

\begin{Shaded}
\begin{Highlighting}[]
\KeywordTok{data}\NormalTok{(}\StringTok{"fibroEset"}\NormalTok{)}
\NormalTok{fib <-}\StringTok{ }\KeywordTok{exprs}\NormalTok{(fibroEset)}
\end{Highlighting}
\end{Shaded}

\subsubsection{2. Select a random set of 50 genes from the data frame,
and subset the data
frame.}\label{select-a-random-set-of-50-genes-from-the-data-frame-and-subset-the-data-frame.}

\begin{Shaded}
\begin{Highlighting}[]
\NormalTok{fib.genes <-}\StringTok{ }\KeywordTok{rownames}\NormalTok{(fib)}
\NormalTok{sample.genes <-}\StringTok{ }\KeywordTok{sample}\NormalTok{(fib.genes, }\DecValTok{50}\NormalTok{)}
\NormalTok{fib.sub <-}\StringTok{ }\NormalTok{fib[sample.genes, ]}
\end{Highlighting}
\end{Shaded}

\newpage

\subsubsection{3. Run and plot hierarchical clustering of the samples
using manhattan distance metric and median linkage method. Make sure
that the sample classification labels are along the x-axis. Title the
plot.}\label{run-and-plot-hierarchical-clustering-of-the-samples-using-manhattan-distance-metric-and-median-linkage-method.-make-sure-that-the-sample-classification-labels-are-along-the-x-axis.-title-the-plot.}

\begin{Shaded}
\begin{Highlighting}[]
\NormalTok{fib.man.samples <-}\StringTok{ }\KeywordTok{dist}\NormalTok{(}\KeywordTok{t}\NormalTok{(fib.sub), }\DataTypeTok{method =} \StringTok{"manhattan"}\NormalTok{)}
\NormalTok{fib.hclust.samples <-}\StringTok{ }\KeywordTok{hclust}\NormalTok{(fib.man.samples, }\DataTypeTok{method =} \StringTok{"median"}\NormalTok{)}
\KeywordTok{plot}\NormalTok{(fib.hclust.samples,}
     \DataTypeTok{main =} \StringTok{"Hierarchical Cluster}\CharTok{\textbackslash{}n}\StringTok{Manhattan"}\NormalTok{,}
     \DataTypeTok{xlab =} \StringTok{"Samples"}\NormalTok{,}
     \DataTypeTok{hang =} \NormalTok{-}\DecValTok{1}\NormalTok{)}
\end{Highlighting}
\end{Shaded}

\includegraphics{lab7-7547611_files/figure-latex/unnamed-chunk-3-1.pdf}

\newpage

\subsubsection{4. Now both run hierarchical clustering and plot the
results in two dimensions (on samples and genes). Plot a heatmap with
the genes on the y-axis and samples on the x-axis. Once again, make sure
that the sample and genes labels are present. Title the
plot.}\label{now-both-run-hierarchical-clustering-and-plot-the-results-in-two-dimensions-on-samples-and-genes.-plot-a-heatmap-with-the-genes-on-the-y-axis-and-samples-on-the-x-axis.-once-again-make-sure-that-the-sample-and-genes-labels-are-present.-title-the-plot.}

\begin{Shaded}
\begin{Highlighting}[]
\NormalTok{hm.col <-}\StringTok{ }\KeywordTok{c}\NormalTok{(}\StringTok{"#FF0000"}\NormalTok{, }\StringTok{"#CC0000"}\NormalTok{, }\StringTok{"#990000"}\NormalTok{, }\StringTok{"#660000"}\NormalTok{, }\StringTok{"#330000"}\NormalTok{, }\StringTok{"#000000"}\NormalTok{,}
            \StringTok{"#000000"}\NormalTok{, }\StringTok{"#0A3300"}\NormalTok{, }\StringTok{"#146600"}\NormalTok{, }\StringTok{"#1F9900"}\NormalTok{, }\StringTok{"#29CC00"}\NormalTok{, }\StringTok{"#33FF00"}\NormalTok{)}
\KeywordTok{heatmap}\NormalTok{(fib.sub, }\DataTypeTok{main =} \StringTok{"Heatmap"}\NormalTok{, }\DataTypeTok{xlab =} \StringTok{"Samples"}\NormalTok{, }\DataTypeTok{ylab =} \StringTok{"Genes"}\NormalTok{, }\DataTypeTok{col =} \NormalTok{hm.col)}
\end{Highlighting}
\end{Shaded}

\includegraphics{lab7-7547611_files/figure-latex/unnamed-chunk-4-1.pdf}

\newpage

\subsubsection{5. Calculate PCA on the samples and retain the first two
components vectors (eigenfunctions). Calculate k-means clustering on
these first two components with
k=3.}\label{calculate-pca-on-the-samples-and-retain-the-first-two-components-vectors-eigenfunctions.-calculate-k-means-clustering-on-these-first-two-components-with-k3.}

\begin{Shaded}
\begin{Highlighting}[]
\NormalTok{fib.pca <-}\StringTok{ }\KeywordTok{prcomp}\NormalTok{(}\DataTypeTok{x =} \KeywordTok{t}\NormalTok{(fib.sub))}
\NormalTok{fib.kmeans <-}\StringTok{ }\KeywordTok{kmeans}\NormalTok{(fib.pca$x[,}\DecValTok{1}\NormalTok{:}\DecValTok{2}\NormalTok{], }\DataTypeTok{centers =} \DecValTok{3}\NormalTok{)}
\KeywordTok{plot}\NormalTok{(fib.sub, }\DataTypeTok{col =} \NormalTok{fib.kmeans$cluster, }\DataTypeTok{cex=}\DecValTok{1}\NormalTok{, }\DataTypeTok{xlab =} \StringTok{"PC1"}\NormalTok{, }\DataTypeTok{ylab =} \StringTok{"PC2"}\NormalTok{,}
     \DataTypeTok{main =} \StringTok{"PCA Plot"}\NormalTok{)}
\KeywordTok{points}\NormalTok{(fib.kmeans$centers, }\DataTypeTok{col =} \DecValTok{1}\NormalTok{:}\DecValTok{4}\NormalTok{, }\DataTypeTok{pch =} \StringTok{"*"}\NormalTok{, }\DataTypeTok{cex =} \FloatTok{2.5}\NormalTok{)}
\end{Highlighting}
\end{Shaded}

\includegraphics{lab7-7547611_files/figure-latex/unnamed-chunk-5-1.pdf}

\newpage

\subsubsection{6. Plot a two-dimensional scatter plot of the sample
classification labels, embedded with the first two eigenfunctions (from
PCA). Color the labels with the color that corresponds to the predicted
cluster membership. Make sure to label the axes and title the plot.
Color based on kmeans cluster. Put the different species and identify
them and then color them based on kmeans cluster to see which species
didn't cluster
correctly.}\label{plot-a-two-dimensional-scatter-plot-of-the-sample-classification-labels-embedded-with-the-first-two-eigenfunctions-from-pca.-color-the-labels-with-the-color-that-corresponds-to-the-predicted-cluster-membership.-make-sure-to-label-the-axes-and-title-the-plot.-color-based-on-kmeans-cluster.-put-the-different-species-and-identify-them-and-then-color-them-based-on-kmeans-cluster-to-see-which-species-didnt-cluster-correctly.}

\begin{Shaded}
\begin{Highlighting}[]
\NormalTok{fib.species <-}\StringTok{ }\NormalTok{fibroEset$species}
\KeywordTok{plot}\NormalTok{(fib.sub, }\DataTypeTok{col =} \NormalTok{fib.kmeans$cluster, }\DataTypeTok{cex =} \DecValTok{1}\NormalTok{, }\DataTypeTok{xlab =} \StringTok{"PC1"}\NormalTok{, }\DataTypeTok{ylab =} \StringTok{"PC2"}\NormalTok{,}
     \DataTypeTok{main =} \StringTok{"PCA Plot"}\NormalTok{)}
\KeywordTok{text}\NormalTok{(fib.sub, }\DataTypeTok{labels =} \NormalTok{fib.species, }\DataTypeTok{cex =} \DecValTok{1}\NormalTok{, }\DataTypeTok{pos =} \DecValTok{1}\NormalTok{)}
\end{Highlighting}
\end{Shaded}

\includegraphics{lab7-7547611_files/figure-latex/unnamed-chunk-6-1.pdf}

\end{document}
