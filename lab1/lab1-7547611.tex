\documentclass[]{article}
\usepackage{lmodern}
\usepackage{amssymb,amsmath}
\usepackage{ifxetex,ifluatex}
\usepackage{fixltx2e} % provides \textsubscript
\ifnum 0\ifxetex 1\fi\ifluatex 1\fi=0 % if pdftex
  \usepackage[T1]{fontenc}
  \usepackage[utf8]{inputenc}
\else % if luatex or xelatex
  \ifxetex
    \usepackage{mathspec}
  \else
    \usepackage{fontspec}
  \fi
  \defaultfontfeatures{Ligatures=TeX,Scale=MatchLowercase}
  \newcommand{\euro}{€}
\fi
% use upquote if available, for straight quotes in verbatim environments
\IfFileExists{upquote.sty}{\usepackage{upquote}}{}
% use microtype if available
\IfFileExists{microtype.sty}{%
\usepackage{microtype}
\UseMicrotypeSet[protrusion]{basicmath} % disable protrusion for tt fonts
}{}
\usepackage[margin=1in]{geometry}
\usepackage{hyperref}
\PassOptionsToPackage{usenames,dvipsnames}{color} % color is loaded by hyperref
\hypersetup{unicode=true,
            pdftitle={Lab1},
            pdfauthor={Jessica Temporal},
            pdfborder={0 0 0},
            breaklinks=true}
\urlstyle{same}  % don't use monospace font for urls
\usepackage{color}
\usepackage{fancyvrb}
\newcommand{\VerbBar}{|}
\newcommand{\VERB}{\Verb[commandchars=\\\{\}]}
\DefineVerbatimEnvironment{Highlighting}{Verbatim}{commandchars=\\\{\}}
% Add ',fontsize=\small' for more characters per line
\usepackage{framed}
\definecolor{shadecolor}{RGB}{248,248,248}
\newenvironment{Shaded}{\begin{snugshade}}{\end{snugshade}}
\newcommand{\KeywordTok}[1]{\textcolor[rgb]{0.13,0.29,0.53}{\textbf{{#1}}}}
\newcommand{\DataTypeTok}[1]{\textcolor[rgb]{0.13,0.29,0.53}{{#1}}}
\newcommand{\DecValTok}[1]{\textcolor[rgb]{0.00,0.00,0.81}{{#1}}}
\newcommand{\BaseNTok}[1]{\textcolor[rgb]{0.00,0.00,0.81}{{#1}}}
\newcommand{\FloatTok}[1]{\textcolor[rgb]{0.00,0.00,0.81}{{#1}}}
\newcommand{\ConstantTok}[1]{\textcolor[rgb]{0.00,0.00,0.00}{{#1}}}
\newcommand{\CharTok}[1]{\textcolor[rgb]{0.31,0.60,0.02}{{#1}}}
\newcommand{\SpecialCharTok}[1]{\textcolor[rgb]{0.00,0.00,0.00}{{#1}}}
\newcommand{\StringTok}[1]{\textcolor[rgb]{0.31,0.60,0.02}{{#1}}}
\newcommand{\VerbatimStringTok}[1]{\textcolor[rgb]{0.31,0.60,0.02}{{#1}}}
\newcommand{\SpecialStringTok}[1]{\textcolor[rgb]{0.31,0.60,0.02}{{#1}}}
\newcommand{\ImportTok}[1]{{#1}}
\newcommand{\CommentTok}[1]{\textcolor[rgb]{0.56,0.35,0.01}{\textit{{#1}}}}
\newcommand{\DocumentationTok}[1]{\textcolor[rgb]{0.56,0.35,0.01}{\textbf{\textit{{#1}}}}}
\newcommand{\AnnotationTok}[1]{\textcolor[rgb]{0.56,0.35,0.01}{\textbf{\textit{{#1}}}}}
\newcommand{\CommentVarTok}[1]{\textcolor[rgb]{0.56,0.35,0.01}{\textbf{\textit{{#1}}}}}
\newcommand{\OtherTok}[1]{\textcolor[rgb]{0.56,0.35,0.01}{{#1}}}
\newcommand{\FunctionTok}[1]{\textcolor[rgb]{0.00,0.00,0.00}{{#1}}}
\newcommand{\VariableTok}[1]{\textcolor[rgb]{0.00,0.00,0.00}{{#1}}}
\newcommand{\ControlFlowTok}[1]{\textcolor[rgb]{0.13,0.29,0.53}{\textbf{{#1}}}}
\newcommand{\OperatorTok}[1]{\textcolor[rgb]{0.81,0.36,0.00}{\textbf{{#1}}}}
\newcommand{\BuiltInTok}[1]{{#1}}
\newcommand{\ExtensionTok}[1]{{#1}}
\newcommand{\PreprocessorTok}[1]{\textcolor[rgb]{0.56,0.35,0.01}{\textit{{#1}}}}
\newcommand{\AttributeTok}[1]{\textcolor[rgb]{0.77,0.63,0.00}{{#1}}}
\newcommand{\RegionMarkerTok}[1]{{#1}}
\newcommand{\InformationTok}[1]{\textcolor[rgb]{0.56,0.35,0.01}{\textbf{\textit{{#1}}}}}
\newcommand{\WarningTok}[1]{\textcolor[rgb]{0.56,0.35,0.01}{\textbf{\textit{{#1}}}}}
\newcommand{\AlertTok}[1]{\textcolor[rgb]{0.94,0.16,0.16}{{#1}}}
\newcommand{\ErrorTok}[1]{\textcolor[rgb]{0.64,0.00,0.00}{\textbf{{#1}}}}
\newcommand{\NormalTok}[1]{{#1}}
\usepackage{graphicx,grffile}
\makeatletter
\def\maxwidth{\ifdim\Gin@nat@width>\linewidth\linewidth\else\Gin@nat@width\fi}
\def\maxheight{\ifdim\Gin@nat@height>\textheight\textheight\else\Gin@nat@height\fi}
\makeatother
% Scale images if necessary, so that they will not overflow the page
% margins by default, and it is still possible to overwrite the defaults
% using explicit options in \includegraphics[width, height, ...]{}
\setkeys{Gin}{width=\maxwidth,height=\maxheight,keepaspectratio}
\setlength{\parindent}{0pt}
\setlength{\parskip}{6pt plus 2pt minus 1pt}
\setlength{\emergencystretch}{3em}  % prevent overfull lines
\providecommand{\tightlist}{%
  \setlength{\itemsep}{0pt}\setlength{\parskip}{0pt}}
\setcounter{secnumdepth}{0}

%%% Use protect on footnotes to avoid problems with footnotes in titles
\let\rmarkdownfootnote\footnote%
\def\footnote{\protect\rmarkdownfootnote}

%%% Change title format to be more compact
\usepackage{titling}

% Create subtitle command for use in maketitle
\newcommand{\subtitle}[1]{
  \posttitle{
    \begin{center}\large#1\end{center}
    }
}

\setlength{\droptitle}{-2em}
  \title{Lab1}
  \pretitle{\vspace{\droptitle}\centering\huge}
  \posttitle{\par}
  \author{Jessica Temporal}
  \preauthor{\centering\large\emph}
  \postauthor{\par}
  \predate{\centering\large\emph}
  \postdate{\par}
  \date{August 4, 2016}



% Redefines (sub)paragraphs to behave more like sections
\ifx\paragraph\undefined\else
\let\oldparagraph\paragraph
\renewcommand{\paragraph}[1]{\oldparagraph{#1}\mbox{}}
\fi
\ifx\subparagraph\undefined\else
\let\oldsubparagraph\subparagraph
\renewcommand{\subparagraph}[1]{\oldsubparagraph{#1}\mbox{}}
\fi

\begin{document}
\maketitle

Download alon.txt:
\url{https://drive.google.com/open?id=0B0-8N2fjttG-Xy1sQUNRREk2RUk} and
read \texttt{alon.txt} data

\begin{Shaded}
\begin{Highlighting}[]
\NormalTok{alon.data =}\StringTok{ }\KeywordTok{read.table}\NormalTok{(}\StringTok{"~/Documents/methods_bioinformatics/data/lab1/alon.txt"}\NormalTok{,}
                       \DataTypeTok{header =} \NormalTok{T)}
\end{Highlighting}
\end{Shaded}

Often in R, our data frames are read in with the gene names as a data
column, instead of a row name. By doing the previous step, we are
removing the gene names from a data column and setting them to the row
names. (Hint: use dimnames(x){[}{[}1{]}{]} on the left side of the
assignment and cast the first column to character (as.character()) prior
to setting the row names).

Setting the row names to the first column, then removing this first
column.

\begin{Shaded}
\begin{Highlighting}[]
\KeywordTok{rownames}\NormalTok{(alon.data) <-}\StringTok{ }\KeywordTok{as.character}\NormalTok{(alon.data[,}\DecValTok{1}\NormalTok{])}
\NormalTok{alon.data$Gene <-}\StringTok{ }\OtherTok{NULL}
\end{Highlighting}
\end{Shaded}

There should be 62 samples. If you have 63 samples, you still have the
row names in the first data column. Looking at the dimensions of the
data.

\begin{Shaded}
\begin{Highlighting}[]
\KeywordTok{dim}\NormalTok{(alon.data)}
\end{Highlighting}
\end{Shaded}

\begin{verbatim}
## [1] 2000   62
\end{verbatim}

Print the sample names to screen.

\begin{Shaded}
\begin{Highlighting}[]
\KeywordTok{names}\NormalTok{(alon.data)}
\end{Highlighting}
\end{Shaded}

\begin{verbatim}
##  [1] "norm1"   "norm2"   "norm3"   "norm4"   "norm5"   "norm6"   "norm7"  
##  [8] "norm8"   "norm9"   "norm10"  "norm11"  "norm12"  "norm13"  "norm14" 
## [15] "norm15"  "norm16"  "norm17"  "norm18"  "norm19"  "norm20"  "norm21" 
## [22] "norm22"  "tumor1"  "tumor2"  "tumor3"  "tumor4"  "tumor5"  "tumor6" 
## [29] "tumor7"  "tumor8"  "tumor9"  "tumor10" "tumor11" "tumor12" "tumor13"
## [36] "tumor14" "tumor15" "tumor16" "tumor17" "tumor18" "tumor19" "tumor20"
## [43] "tumor21" "tumor22" "tumor23" "tumor24" "tumor25" "tumor26" "tumor27"
## [50] "tumor28" "tumor29" "tumor30" "tumor31" "tumor32" "tumor33" "tumor34"
## [57] "tumor35" "tumor36" "tumor37" "tumor38" "tumor39" "tumor40"
\end{verbatim}

Plotting one of the tumor samples versus one of the normal samples in an
xy scatter plot. Remember that the first argument is the x vector. Label
the x and y-axes as `normal' and `tumor', respectively. Title the plot,
`Tumor sample vs.~Normal sample - 2000 genes'.

\begin{Shaded}
\begin{Highlighting}[]
\KeywordTok{plot}\NormalTok{(alon.data$norm1, alon.data$tumor1, }\DataTypeTok{xlab =} \StringTok{"normal"}\NormalTok{, }\DataTypeTok{ylab =} \StringTok{"tumor"}\NormalTok{,}
     \DataTypeTok{main =} \StringTok{"Tumor sample vs. Normal sample - 2000 genes"}\NormalTok{)}
\end{Highlighting}
\end{Shaded}

\includegraphics{lab1-7547611_files/figure-latex/unnamed-chunk-4-1.pdf}

Now do the same with 2 normal samples, adjusting the axes labels and
title, but pick only 20 genes.

\begin{Shaded}
\begin{Highlighting}[]
\NormalTok{norm1and2 =}\StringTok{ }\KeywordTok{cbind}\NormalTok{(alon.data$norm1[}\DecValTok{1}\NormalTok{:}\DecValTok{20}\NormalTok{], alon.data$norm2[}\DecValTok{1}\NormalTok{:}\DecValTok{20}\NormalTok{])}
\NormalTok{tumor1an2 =}\StringTok{ }\KeywordTok{cbind}\NormalTok{(alon.data$tumor1[}\DecValTok{1}\NormalTok{:}\DecValTok{20}\NormalTok{], alon.data$tumor2[}\DecValTok{1}\NormalTok{:}\DecValTok{20}\NormalTok{])}
\KeywordTok{plot}\NormalTok{(norm1and2, tumor1an2, }\DataTypeTok{xlab =} \StringTok{"normal"}\NormalTok{, }\DataTypeTok{ylab =} \StringTok{"tumor"}\NormalTok{,}
     \DataTypeTok{main =} \StringTok{"Tumor sample vs. Normal sample - 20 genes"}\NormalTok{)}
\end{Highlighting}
\end{Shaded}

\includegraphics{lab1-7547611_files/figure-latex/unnamed-chunk-5-1.pdf}

Add a line to connect the points

\begin{Shaded}
\begin{Highlighting}[]
\KeywordTok{plot}\NormalTok{(norm1and2, tumor1an2, }\DataTypeTok{xlab =} \StringTok{"normal"}\NormalTok{, }\DataTypeTok{ylab =} \StringTok{"tumor"}\NormalTok{,}
     \DataTypeTok{main =} \StringTok{"Tumor sample vs. Normal sample - 20 genes"}\NormalTok{)}
\KeywordTok{lines}\NormalTok{(norm1and2, tumor1an2)}
\end{Highlighting}
\end{Shaded}

\includegraphics{lab1-7547611_files/figure-latex/unnamed-chunk-6-1.pdf}

Take the ratio of gene 5 to gene 15 and plot the profile of the gene
across all samples. Label each point with the sample name (see text()
help and use cex=1).

\begin{Shaded}
\begin{Highlighting}[]
\NormalTok{ratio5and15 <-}\StringTok{ }\NormalTok{alon.data[}\DecValTok{5}\NormalTok{,]/alon.data[}\DecValTok{15}\NormalTok{,]}
\KeywordTok{plot}\NormalTok{(}\DecValTok{1}\NormalTok{:}\DecValTok{62}\NormalTok{, ratio5and15, }\DataTypeTok{xlab =} \StringTok{"samples"}\NormalTok{, }\DataTypeTok{main =} \StringTok{"Gene profile across samples"}\NormalTok{)}
\KeywordTok{text}\NormalTok{(}\DecValTok{1}\NormalTok{:}\DecValTok{62}\NormalTok{, ratio5and15, }\DataTypeTok{labels =} \KeywordTok{names}\NormalTok{(alon.data), }\DataTypeTok{cex =} \DecValTok{1}\NormalTok{)}
\end{Highlighting}
\end{Shaded}

\includegraphics{lab1-7547611_files/figure-latex/unnamed-chunk-7-1.pdf}

\end{document}
